\documentclass[10pt]{article}
\usepackage[russian]{babel}
\usepackage[utf8]{inputenc}
\usepackage{amssymb}
\usepackage{amsmath}
\usepackage{latexsym}
\usepackage{enumitem}
\usepackage[margin=2cm]{geometry}
\usepackage{tikz}
\usepackage{graphicx}
\usetikzlibrary{arrows}

\renewcommand{\leq}{\leqslant}
\renewcommand{\geq}{\geqslant}

\begin{document}

\title{Домашняя работа 6}
\author{Антон Афанасьев}
\maketitle

\begin{enumerate}
\item[1] Представим $k$-мерный гиперкуб как два $(k-1)$-мерных гиперкуба, у которых соответствующие вершины соединены ребрами. Для вершины $v$ обозначим соответствующую ей вершину из ``параллельного'' гиперкуба как $v'$. 

Докажем $k$-связность по индукции. Для $k=1$ выполняется. Пусть выполняется для $k-1$. Тогда для $k$-мерного гиперкуба:

Возьмем две вершины $u$ и $v$ из одного ``подкуба'' размерности $k-1$. Тогда, по предположению индукции и теореме Уитни между ними есть $k-1$ путь, не имеющих общих внутренних вершин. Построим еще один путь через параллельный гиперкуб (перейдя $u \to u' \leadsto v' \to v$). Теперь между этими вершинами есть $k$ путей, не имеющих общих внутренних вершин.

Возьмем две вершины $u$ и $v'$ из параллельных гиперкубов. Тогда, по предположению индукции  и теореме Уитни есть $k-1$ путь, не имеющих общих внутренних вершин, из $u$ в $v$ и из $u'$ в $v'$. Выберем пути так, чтобы в обоих подкубах они были одинаковые и обозначим их как $p_1, \ldots p_{k-1}$ и $p'_1, \ldots p'_{k-1}$. Выберем $k$ путей между $u$ и $v'$ следующим образом: 
\begin{itemize}
\item из $k-2$ соседей $u$ из того же подкуба перейдем в параллельный куб, и из соответствующих им вершин перейдем в $v'$ по путям $p'_1, \ldots p'_{k-2}$.
\item из $u$ перейдем в $u'$, а потом по пути $p'_{k-1}$ в $v'$.
\item из $u$ по пути $p_{k-1}$ перейдем в $v$, затем в $v'$.
\end{itemize}
Мы получили $k$ путей, не имеющих общих внутренних вершин.

Итак, для любой пары вершин, между ними существует $k$ путей не имеющих общих внутренних вершин, значит, по теореме Уитни, $k$-мерный гиберкуб $k$-связен.

\end{enumerate}

\end{document}
