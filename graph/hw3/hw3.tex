\documentclass[10pt]{article}
\usepackage[russian]{babel}
\usepackage[utf8]{inputenc}
\usepackage{amssymb}
\usepackage{amsmath}
\usepackage{latexsym}
\usepackage{enumitem}
\usepackage[margin=2cm]{geometry}
\usepackage{tikz}
\usepackage{graphicx}
\usetikzlibrary{arrows}

\renewcommand{\leq}{\leqslant}
\renewcommand{\geq}{\geqslant}

\begin{document}

\title{Домашняя работа 3}
\author{Антон Афанасьев}
\maketitle

\begin{enumerate}

\item[1.17.] Пусть у графа $G$ диаметр больше трех. Рассмотрим его дополнение $\overline G$. Если между вершинами $x,\ y$ в $G$ не было ребра, то в $\overline G$ ребро будет, а значит, расстояние между ними $d(x, y)=1$. Пусть ребро $(x, y)$ есть, тогда найдется вершина $v \in V(G)$, такая что она не смежна ни с $x$ ни с $y$. Иначе все вершины смежны либо с $x$ либо с $y$, а значит между всеми вершинами есть путь длины меньше, либо равной трем, что противоречит тому, что диаметр $G$ больше трех. Значит, в $\overline G$ между $x$ и $y$ будет путь длины 2 через $u$.\\
Следовательно, диаметр дополнения графа $G$ меньше трех.

\item[3.15.] $\Rightarrow$

Пусть в $G$ существует гамильтонов цикл, тогда он существует и в $G+\{x,y\}$, так как добавление ребра ничего не испортит.

$\Leftarrow$

Пусть в $G+\{x,y\}$ существует гамильтонов цикл. Если он не проходит по ребру $(x, y)$, то после удаления этого ребра он так же останется гамильтоновым циклом в графе $G$. Пусть цикл проходит по ребру $(x, y)$, тогда после удаления ребра у нас останется гамильтонов путь, с крайними вершинами $x,\ y$. По лемме 3.8 в таком графе есть гамильтонов цикл.

\item[4.3.] Рассмотрим раскраску графа $G$ в $\chi(G)$ цветов. Обозначим количество вершин, покрашенных в цвет $c$, как $cnt_c$. Заметим, что подмножества вершин, покрашенных в один цвет, являются независимыми подмножествами. Следовательно, $cnt_c \leq \alpha(G)$. $$n = \sum_{c=1}^{\chi(G)} cnt_c \leq \sum_{c=1}^{\chi(G)} \alpha(G) \leq \chi(G) \cdot \alpha(G)$$
Что и требовалось доказать.

\item[4.5.] Докажем, что $\chi(G) \cdot \chi(\overline G) \geq n$.\\
Рассмотрим раскраску графа $G$ в $\chi(G)$ цветов. Обозначим количество вершин, покрашенных в цвет $c$, как $cnt_c$. Так как множества вершин одного цвета --- независимые множества, то в $\overline G$ эти вершины будут образовывать клику. И наоборот, клика в $\overline G$ будет независимым множеством в $G$. Из этого следует, что $\alpha(G) = \omega(\overline G)$. Т.к. $\omega(\overline G) \leq \chi(\overline G)$, то $cnt_c \leq \alpha(G) = \omega(\overline G) \leq \chi(\overline G)$. Следовательно, $cnt_c \leq \chi(\overline G)$. Сделав такой же переход, как в предыдущем задании, увидим, что $\chi(G) \cdot \chi(\overline G) \geq n$.

\item[4.8.] Из того, что в графе $G$ нет маршрутов длины, большей, чем $m-1$, следует, что все маршруты в этом графе являются простыми путями (иначе был бы цикл, по которому можно ходить сколько угодно и делать маршруты какой угодно длины). Построим раскраску, которая будет использовать не более $m$ цветов. Покрасим все вершины, у которых нет исходящих ребер в первый цвет. Теперь каждую вершину $v$, у которой все соседние вершины $u$ в которые есть ребро из $v$ уже покрашены, красим в цвет с номером на 1 больше чем максимальный из цветов $u$. Такая раскраска будет корректной (каждое ребро соединяет вершины разного цвета). И она использует не более $m$ цветов, т.к. цвета увеличиваются вдоль какого-то маршрута (по которому идем в обратном порядке). Если бы это было не так, то можно было бы пройти по убыванию цветов из вершины с максимальным цветом, и получить маршрут длины больше чем $m-1$.\\
Получили раскраску, которая использует не более $m$ цветов, следовательно $\chi(G) \leq m$.

\item[4.9.] 
	\begin{itemize}
		\item $x_1, x_2, \ldots, x_n, y_1, y_2, \ldots, y_n$
		
		Сначала для всех $x_i$ будет выбираться первый цвет, так как нет ни одной смежной покрашенной вершины, а потом для всех $y_i$ будет выбираться второй цвет, поскольку все смежные вершины окрашены в первый.
		\item $x_1, y_1, x_2, y_2, \ldots, x_n, y_n$
		
		Вершины $x_i, y_i$ будут краситься в минимальный еще не использованный цвет.
	\end{itemize}

\item[4.10.] Рассмотрим раскраску в $\chi(G)$ цветов. Перекрасим вершины так, чтобы каждая вершина цвета не 1 была смежна с какой-либо вершиной цвета 1. Для этого просто пройдем по вершинам и перекрасим ее в цвет 1, если это возможно. Раскраска не перестанет от этого быть корректной и число цветов не изменится. Теперь возьмем все вершины цвета, отличного от 1, и проведем с ними такую процедуру, только теперь с цветом 2 (чтобы каждая вершина цвета не 2 была смежна с вершиной цвета 2). Продолжим так делать для всех цветов.

Теперь упорядочим вершины следующим образом --- сначала вершины цвета 1, потом цвета 2 и т.д. Жадный алгоритм покрасит эти вершины точно так же, т.к. вершины цвета 1 между собой не смежны, а каждая из остальных вершин смежна с хотя бы одной вершиной каждого меньшего цвета.

\item[4.17.] Пусть $a_n = |V(G_{n+2})|$, $a_0 = |V(G_2)| = 2$

$a_{n+1} = 2a_n +1$

Догадка: $a_n = 2^{n+1} + 2^n - 1 = 3\cdot 2 ^n - 1$. Докажем, что это так.

$a_0 = 3 \cdot 2^0 - 1 = 2$\\
$a_{n+1} = 2(3\cdot 2^n - 1) + 1 = 3 \cdot 2^{n+1} -2 +1 = 3 \cdot 2^{n+1} - 1$

Догадка верна, а значит $|V(G_k)| = a_{k-2} = 3 \cdot 2^{k-2}-1$
\end{enumerate}

\end{document}
