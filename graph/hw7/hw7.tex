\documentclass[10pt]{article}
\usepackage[russian]{babel}
\usepackage[utf8]{inputenc}
\usepackage{amssymb}
\usepackage{amsmath}
\usepackage{latexsym}
\usepackage{enumitem}
\usepackage[margin=2cm]{geometry}
\usepackage{tikz}
\usepackage{graphicx}
\usetikzlibrary{arrows}

\renewcommand{\leq}{\leqslant}
\renewcommand{\geq}{\geqslant}

\begin{document}

\title{Домашняя работа 7}
\author{Антон Афанасьев}
\maketitle

\begin{enumerate}
\item[7.2] Так как каждая вершина из первой доли связанна с каждой вершиной второй доли, то ответ --- число инъекций. Ответ: $(m)_n$ 

\item[7.4] На рисунке ребро (4, 2) составляет наибольшее по включению паросочетание, но максимальным является паросочетание из ребер (1,2) и (4, 3).

\begin{tikzpicture}
  [scale=0.5, auto=left,every node/.style={circle,fill=blue!20}]
  \tikzstyle{selected edge} = [draw,line width=1pt,-,green]	
  \node (n1) at (1,1) {1};
  \node (n2) at (5,1)  {2};
  \node (n3) at (5,5)  {3};
  \node (n4) at (1,5) {4};

  \foreach \from/\to in {n1/n2,n4/n3}
    \draw (\from) -- (\to);
    
	\draw[selected edge] (n4) -- (n2);
\end{tikzpicture}

\item[7.7] Построим двудольный граф. $m$ вершин в первой доле будут соответствовать мастям карт, $m$ вершин во второй доле будут соответствовать столбцам таблицы. Проведем ребро между вершинами $x_i$ и $y_j$,  если в столбце $j$ есть карта масти $i$. Тогда, если в таком графе существует $X$-насыщенное паросочетание, то это значит, что есть $m$ карт разной масти, которые лежат в разных столбцах.

Любому подмножеству первой доли размером $k$ соответствует $n \cdot k$ карт. Эти карты разложены как минимум в $k$ столбцов, а значит, вершины из этого подмножества смежны как минимум с $k$ вершинами второй доли. Тогда, по теореме Холла, $X$-насыщенное паросочетакие существует.

\end{enumerate}

\end{document}
