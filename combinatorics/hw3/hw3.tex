\documentclass[10pt]{article}
\usepackage[russian]{babel}
\usepackage[utf8]{inputenc}
\usepackage{amssymb}
\usepackage{amsmath}
\usepackage{latexsym}
\usepackage{enumitem}
\usepackage[margin=2cm]{geometry}

\newcommand{\rchoose}[2]{\left(\mkern-6mu \left({#1 \atop #2}\right) \mkern-6mu \right)}

\begin{document}

\title{Домашняя работа 3}
\author{Антон Афанасьев}
\maketitle

\begin{enumerate}[label*=4.\arabic*]
	\item[4.4] Ответ --- число сюръекций из 7-ми элементного множества групп в 5-ти элементное множество преподавателей: $\hat S (7, 5)$.
	
	\item[4.5] Выберем сначала $r$ ящиков, в которые будем класть $n$ предметов, потом положим.\\
	Ответ: $ \binom{k}{r} \cdot r^n$
	
	\item[4.6]
	\begin{itemize}
		\item[а)] $\frac{10!}{2! \cdot 3! \cdot 2!}$
		\item[б)] $\frac{13!}{2! \cdot 2! \cdot 2! \cdot 2!}$
	\end{itemize}
	
	\item[4.7]
	\begin{itemize}
		\item[а)] Для какого-то одного разряда каждая цифра будет встречаться $3!$ раз. Сумма цифр в каждом разряде будет $3! \cdot (1+2+3+4) = 60$. Осталось сложить сумму по разрядам: $60 \cdot (1 + 10 + 10^2 +10^3) = 66660$\\
		Ответ: 66660
		\item[б)] В каждом разряде цифры 1 и 5 будут встречаться $\frac{3!}{2!} = 3$ раза, а 2 --- $3! = 6$ раз. Сумма цифр по разрядам $5 \cdot 3 + 1 \cdot 3 + 2 \cdot 6 = 30$.\\
		Ответ: $30 \cdot 1111 = 33330$
	\end{itemize}
	
	\item[4.8] 
	$$\sum_{a_1+a_2+a_3+a_4+a_5 = 20} P(20; a_1, a_2, a_3, a_4, a_5) \cdot a_1! \cdot a_2! \cdot a_3! \cdot a_4! \cdot a_5! = \sum_{a_1+a_2+a_3+a_4+a_5 = 20} 20! = 20! \cdot \rchoose{5}{20}$$
	
	\item[4.9] Ответ: $P(60; 15, 15, 15, 15) = \frac{60!}{(15!)^4}$
	
	\item[4.10] $S(n, 1) = 1$\\
		$S(n, n) = 1$\\
		$S(n, n-1) = \binom{n}{2} = \frac{n(n-1)}{2}$
\end{enumerate}
\end{document}
