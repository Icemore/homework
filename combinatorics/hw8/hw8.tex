\documentclass[10pt]{article}
\usepackage[russian]{babel}
\usepackage[utf8]{inputenc}
\usepackage{amssymb}
\usepackage{amsmath}
\usepackage{latexsym}
\usepackage{enumitem}
\usepackage[margin=2cm]{geometry}
\usepackage{relsize}

\newcommand{\rchoose}[2]{\left(\mkern-6mu \left({#1 \atop #2}\right) \mkern-6mu \right)}
\newcommand{\dsum}{\sum_{n=0}^\infty}
\renewcommand{\P}{\text{Pr}}

\begin{document}

\title{Домашняя работа 8}
\author{Антон Афанасьев}
\maketitle

\begin{enumerate}
\item[11.3] Первое слагаемое --- выпадение бубновой карты, второе -- червовой, третье --- черной.
$$E(\xi) = \sum_{i=1}^{13} \frac{1}{52} \cdot i + \sum_{i=1}^{13} \frac{1}{52} \cdot 2\cdot i + \frac{1}{2} \cdot (-10) = \frac{1}{52} \cdot 91 + \frac{1}{52} \cdot 182 - 5 = \frac{13}{52} = \frac{1}{4}$$

\item[11.9] Числа на второй кубик: 1, 2, 2, 3, 3, 4.

\end{enumerate}
\end{document}
