\documentclass[10pt]{article}
\usepackage[russian]{babel}
\usepackage[utf8]{inputenc}
\usepackage{amssymb}
\usepackage{amsmath}
\usepackage{latexsym}
\usepackage{enumitem}
\usepackage[margin=2cm]{geometry}
\usepackage{relsize}

\newcommand{\rchoose}[2]{\left(\mkern-6mu \left({#1 \atop #2}\right) \mkern-6mu \right)}
\newcommand{\dsum}{\sum_{n=0}^\infty}

\begin{document}

\title{Домашняя работа 6}
\author{Антон Афанасьев}
\maketitle

\begin{enumerate}
	\item[8.8]
	\begin{itemize}
		\item $a_{n+1} = 2(n+1)a_n + (n+1)!$, $a_0 = 0$
		
		Домножим уравнение на $\frac{z^{n+1}}{(n+1)!}$ и просуммируем.
		$$\sum_{n=0}^\infty a_{n+1} \frac{z^{n+1}}{(n+1)!} = 2z \sum_{n=0}^\infty a_n \frac{z^n}{n!} + \sum_{n=0}^\infty z^n$$
		$$F(z) - a_0 = 2zF(z) +\frac{z}{1-z}$$
		$$F(z) = \frac{z}{(1-2z)(1-z)} = \frac{1}{1-2z} - \frac{1}{1-z} = \sum_{n=0}^\infty (2z)^n - \sum_{n=0}^\infty z^n$$
		
		Ответ: $a_n = n!(2^n - 1)$
		
		\item $a_{n+1} = a_{n+1} + (n+1)a_n$, $a_0 = a_1 = 1$
		
		Домножим уравнение на $\frac{z^{n+1}}{(n+1)!}$ и просуммируем.
		
		$$\dsum a_{n+2} \frac{z^{n+1}}{(n+1)!} = \dsum a_{n+1} \frac{z^{n+1}}{(n+1)!} + z \dsum a_n \frac{z^n}{n!}$$
		$F'(z) - a_1 = F(z) - a_0 + zF(z)$\\
		$F'(z) = F(z)(1+z)$\\
		$\mathlarger{\frac{dF}{dz} = F(z)(1+z)}$\\
		$\mathlarger{\frac{dF}{F(z)} = (1+z)dz}$\\
		$d\ln F(z) = d(z+\frac{1}{2}z^2)$\\
		$\ln F(z) = z+\frac{1}{2}z^2 + C$
		
		Так как $F(0)=a_0 = 1$, $\ln F(0) = 0$, то $C = 0$.
		
		$\mathlarger{\mathlarger{F(z) = e^z e^{\frac{1}{2}z^2}}}$
		$$e^z = 1 + \frac{z}{1!} + \frac{z^2}{2!} + \frac{z^3}{3!} + \ldots$$
		$$e^{(1/2)z^2} = 1 + \frac{1}{1!} \frac{1}{2} z^2 + \frac{1}{2!} \frac{1}{2^2} z^4 + \frac{1}{3!} \frac{1}{2^3} z^6 + \ldots = \dsum \frac{(2n)_n}{2^n} \frac{z^{2n}}{(2n)!}$$
		
		Перемножаем эти ряды и получаем, что
		
		$$a_n = \sum_{i=0}^{\left \lfloor \frac{n}{2} \right \rfloor} \binom{n}{2i} \frac{(2i)_i}{2^i}$$
		
		\item $a_n = na_{n-1} + n(n-1)a_{n-2}$, $a_0 = a_1 = 1$\\
		$a_{n+2} = (n+2)a_{n+1} + (n+2)(n+1)a_{n}$, $a_0 = a_1 = 1$
		
		Домножим уравнение на $\frac{z^{n+2}}{(n+2)!}$ и просуммируем.
		$$\dsum a_{n+2} \frac{z^{n+2}}{(n+2)!} = z \dsum a_{n+1} \frac{z^{n+1}}{(n+1)!} + z^2 \dsum a_n \frac{z^n}{n!}$$
		$$F(z) - a_0 - a_1z = z(F(z) - a_0) + z^2 F(z)$$
		$$F(z) = \frac{1}{1-z-z^2}$$
		
		Заметим, что $zF(z) = \frac{z}{1-z-z^2}$ --- обыкновенная производящая функция для чисел Фибоначчи. Значит,\\
		$a_n = n! F_{n+1}$
	\end{itemize}
	
	\item[9.1] Сопоставим шагу вправо открывающую скобку, шагу вверх --- закрывающую. Так как пути не поднимаются выше диагонали, то в любой момент времени $y \le x$, по определению такая последовательность является правильной скобочной последовательностью.
	
	\item[9.7] Рассмотрим число $M_n$. Переберем точку, в которой путь в первый раз вернется на ось абсцисс. Пусть это точка $(i, 0)$. Если $i=1$, то искомое число путей в этом случае равно $M_{n-1}$. Если $i>1$, то путь обязательно проходит через точки $(1, 1)$ и $(i-1, 1)$. Путь между этими точками имеет длину $i-2$ не опускался ниже прямой $y=1$, а значит, число путей между этими точками равно $M_{i-2}$. На путь после $(i, 0)$ не накладывается никаких дополнительных ограничений, и их просто $M_{n-i}$.
	
	Получаем рекуррентное соотношение:
	$$M_n = M_{n-1} + \sum_{i=2}^n M_{i-2} M_{n-i},\ M_0 = 1, M_1 = 1$$
	
	\item[9.8] Сопоставим переходу $(1, 1)$ открывающую скобку, переходу $(1, -1)$ --- закрывающую. Так как путь не опускается ниже оси абсцисс, такая скобочная последовательность является правильной, и число путей состоящих только из таких переходов равно $C_{n/2}$. Пусть теперь пути могут состоять еще из переходов $(1, 0)$. Зафиксируем скобочную последовательность из $k$ пар скобок. Ей соответствует некоторая последовательность переходов вверх и вниз. Заметим, что если теперь мы выберем, в каких точках делать переходы мы однозначно зададим путь. Так как переходов $2k$ количество способов выбрать точки, в которых будут переходы равно $\binom{n}{2k}$. Теперь переберем $k$ и получим искомую формулу:
	$$M_n = \sum_{k=0}^{[n/2]} C_k \binom{n}{2k}$$
\end{enumerate}
\end{document}
