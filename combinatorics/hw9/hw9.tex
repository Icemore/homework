\documentclass[10pt]{article}
\usepackage[russian]{babel}
\usepackage[utf8]{inputenc}
\usepackage{amssymb}
\usepackage{amsmath}
\usepackage{latexsym}
\usepackage{enumitem}
\usepackage[margin=2cm]{geometry}
\usepackage{relsize}

\newcommand{\rchoose}[2]{\left(\mkern-6mu \left({#1 \atop #2}\right) \mkern-6mu \right)}
\newcommand{\dsum}{\sum_{n=0}^\infty}
\renewcommand{\P}{\text{Pr}}

\begin{document}

\title{Домашняя работа 9}
\author{Антон Афанасьев}
\maketitle

\begin{enumerate}
\item[12.1] По определению марковская цепь --- последовательность случайных величин, в которой величины зависят только от предыдущей.
$\P(\xi_n = j | \xi_0 = l, \ldots , \xi_{n-1}=i) = \P(\xi = j | \xi_{n-1} = i)$. Для того, чтобы величины были независимы, нужно, чтобы каждая величина не зависела от предыдущей, то есть, чтобы то в какой позиции мы находимся на $n$-ом шаге не зависело от того, где мы находимся на $n-1$ шаге. Для этого, для каждого состояния вероятности переходов в него из всех других состояний должны быть одинаковы. В матрице перехода в этом случае все строки будут одинаковые.

\item[12.4] Так как каждый 15-минутный отрезок независим от другого, то для одного двигателя вероятность выжить в течение часа равна $p = 0.75^4$. Тогда вероятность не пережить час равна $q = 1-p$.

Для того, чтобы самолет не вернулся на базу у него должно было либо не остаться ни одного двигателя, либо остаться только один. Так как двигатели отказывают независимо, вероятность этого события: $q^4 + 4 \cdot q^3 \cdot p$.

Тогда самолет вернется на базу с вероятностью $1 - (q^4 + 4 \cdot q^3 \cdot p) = 0.3773344 \approx 37.73\% $

\item[12.5] Подсчитаем матожидание выигрыша на $n$-ом шаге.

$$E_n (\xi) = \frac{1}{6} \cdot \frac{2}{5} E_{n-1}(\xi) \left ( 0 + 1 + 2 + 3 + 4 + 5 \right ) = E_{n-1}(\xi)$$

Так как $E_0(\xi) = A$, то $E_n(\xi) = A$.

Для дисперсии подсчитаем таким же образом $E_n (\xi^2)$:
$$E_n(\xi^2) = \frac{1}{6} \cdot \left ( \frac{2}{5} \right )^2 E_{n-1}(\xi^2) \left ( 0^2 + 1^2 + 2^2 + 3^2 + 4^2 + 5^2 \right ) = \frac{22}{15} E_{n-1}(\xi^2)$$

$E_0(\xi^2) = A^2$, значит, $E_n(\xi^2) = \left(\frac{22}{15} \right ) ^n A^2$.

Так как $Var(\xi) = E(\xi^2) - (E(\xi))^2$, то $Var_n(\xi) = \left(\frac{22}{15} \right ) ^n A^2 - A^2$.
\end{enumerate}
\end{document}
