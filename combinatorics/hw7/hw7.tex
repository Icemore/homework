\documentclass[10pt]{article}
\usepackage[russian]{babel}
\usepackage[utf8]{inputenc}
\usepackage{amssymb}
\usepackage{amsmath}
\usepackage{latexsym}
\usepackage{enumitem}
\usepackage[margin=2cm]{geometry}
\usepackage{relsize}

\newcommand{\rchoose}[2]{\left(\mkern-6mu \left({#1 \atop #2}\right) \mkern-6mu \right)}
\newcommand{\dsum}{\sum_{n=0}^\infty}
\renewcommand{\P}{\text{Pr}}

\begin{document}

\title{Домашняя работа 7}
\author{Антон Афанасьев}
\maketitle

\begin{enumerate}
	\item[10.4] Пусть события $A$ --- первый студент решит задачу, $B$ --- второй студент решит задачу, $C$ --- третий студент решит задачу.\\
	$\P(A) = 0.8,\ \P(B) = 0.7\ \P(C) = 0.6$\\
	$\P(A \cup B \cup C) = \P(A) + \P(B) + \P(C) - \P(A \cap B) - \P(A \cap C) - \P(B \cap C) + \P(A \cap B \cap C) = 0.8 + 0.7 + 0.6 - 0.8 \cdot 0.7 - 0.8 \cdot 0.6 - 0.7 \cdot 0.6 + 0.8\cdot0.7\cdot0.6 = 0.976$
	
	\item[10.5] $A$ --- сумма равна 7.
	$B$ --- сумма нечетная.
	
	$\P(A|B) = \frac{\P(A \cap B)}{\P(B)} = \frac{\P(A)}{\P(B)}$
	
	$\P(A \cap B) = \P(A)$, так как $\P(B|A) = 1$.
	
	$\P(A) = \frac{1}{6}$
	
	$\P(B) = \frac{2+4+6+4+2}{36} = \frac{1}{2}$
	
	$\P(A|B) = \frac{1}{3}$
	
	\item[10.6] $\P(A) = \frac{3}{6} = \frac{1}{2}$
	
	$\P(B) = \frac{3}{6} = \frac{1}{2}$
	
	$\P(A \cap B) = \frac{1}{6}$
	
	$\P(A \cap B) \not = \P(A) \cdot \P(B)$, значит, события зависимы.
	
	\item[10.7] Количество способов набрать сумму $s$ равно
	$\begin{cases}
	180 - (s-1),&s>90\\
	s-1,&s\le90
	\end{cases}$ 
	
	Количество способов набрать сумму больше 140:
	$$\sum_{s=141}^{180} (181-s) = 181 \cdot 40 - \sum_{s=141}^{180} s = 181 \cdot 40 - \frac{141+180}{2} \cdot 40 = 7240 - 6420 = 820$$
	
	Количество способов набрать четную сумму:
	$$\sum_{s=1}^{45}(2s-1) + \sum_{s=46}^{90} (181-2s) = 2 \sum_{s=1}^{45}s - 45 + 45 \cdot 181 - 2 \sum_{s=46}^{90} s = 180 \cdot 45 + (45+1) \cdot 45 - (46+90) \cdot 45 = 90 \cdot 45 = 4050$$
	
	Количество способов набрать четную сумму больше 140:
	$$\sum_{s=71}^{90} (181 - 2s) = 181 \cdot 20 - \sum_{s=71}^{90} s = 3620 - (71+90)\cdot 20 = 3620 - 3220 = 400$$
	
	$\P(A) = \frac{4050}{8100} = \frac{1}{2}$
	
	$\P(B) = \frac{820}{8100}$
	
	$\P(A \cap B) = \frac{400}{8100}$
	
	$\P(A) \cdot P(B) = \frac{410}{8100}$
	
	$\P(A \cap B) \not = \P(A) \cdot \P(B)$, значит, события зависимы.
	
	\item[10.9] $B$ --- студент выпустился. $A_1$ --- сдал с первой попытки, $A_2$ --- сдал со второй, $A_3$ --- не сдал.
	
	$\P(A_1) = 0.3$, $\P(A_2) = 0.5$, $\P(A_3) = 0.2$
	
	$\P(B|A_1) = 0.95$, $\P(B|A_2) = 0.6$, $P(B|A_3)=0.2$
	
	По формуле полной вероятности:\\
	$\P(B) = 0.3 \cdot 0.95 + 0.5 \cdot 0.6 + 0.2 \cdot 0.2 = 0.625$
	
	Итого, выпускаются 62.5\%.
	
	\item[10.11] $B$ --- цель поражена. $A_1$ --- стрелок из первой группы, $A_2$ --- из второй, $A_3$ --- из третьей.
	
	$\P(A_1) = 0.5$, $\P(A_2) = 0.3$, $\P(A_3) = 0.2$
	
	$\P(B|A_1) = 0.8$, $\P(B|A_2) = 0.5$, $\P(B | A_3) = 0.9$
	$$\P(A|B) = \frac{\P(A) \cdot \P(B|A)}{\P(B)}$$
	Так как нас не интересуют точные значения вероятностей, не будем считать $\P(B)$, на сравнение это не повлияет.
	
	$\P(A_1|B) \cdot \P(B) = 0.5 \cdot 0.8 = 0.4$\\
	$\P(A_2|B) \cdot \P(B) = 0.3 \cdot 0.5 = 0.15$\\
	$\P(A_3|B) \cdot \P(B) = 0.2 \cdot 0.9 = 0.18$	\\
	
	Значит, вероятнее всего стрелок был из первой группы.
\end{enumerate}
\end{document}
