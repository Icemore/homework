\documentclass[10pt]{article}
\usepackage[russian]{babel}
\usepackage[utf8]{inputenc}
\usepackage{amssymb}
\usepackage{amsmath}
\usepackage{latexsym}
\usepackage{enumitem}
\usepackage[margin=2cm]{geometry}


\newcommand{\rchoose}[2]{\left(\mkern-6mu \left({#1 \atop #2}\right) \mkern-6mu \right)}

  
\begin{document}

\title{Домашняя работа 2}
\author{Антон Афанасьев}
\maketitle

\begin{enumerate}[label*=3.\arabic*]
	\item $P(n, k) = P(n-1, k) + kP(n-1, k-1)$
	
	Рассмотрим множество $k$-перестановок из $n$ элементов без повторений, их $P(n, k)$. Из этого множества возьмем все перестановки не содержащие элемент $x_n$, их $P(n-1, k)$. Теперь возьмем все перестановки содержащие элемент $x_n$, их $kP(n-1, k-1)$ ($x_n$ уже выбран и одно место уже занято, значит все остальное выбираем $P(n-1, k-1)$ способами, и в каждой такой перестановке $x_n$ можно поставить на одно из $k$ мест). Так как эти подмножества образуют разбиение исходного множества, то его мощность равна сумме мощностей подмножеств.
	
	\item Каждый светофор выбирает одно из трех состояний независимо от других (вытаскивает цвет из урны, а потом кладет его обратно).\\
	$\Rightarrow$ Ответ: $3^6=729$
	
	\item Каждый пассажир независимо выбирает одну остановку на которой будет выходить.\\
	$\Rightarrow$ Ответ: $5^6=15625$
	
	\item Так как предметы неразличимы, раскладки различаются только числом элементов в каждом ящике. Мы можем просто считать, что мы уже положили в каждый ящик по одному предмету, тогда задача эквивалентна задаче разложить оставшиеся, но уже без ограничений на количество предметов в ящике.
	$$\rchoose{n}{k-n} = \binom{k-1}{k-n} = \binom{k-1}{n-1}$$

	\item Применяем ту же логику что в 3.4. Единички которые мы собрались раскладывать неразличимы, поэтому можем просто считать, что на позиции $i$ уже лежит $s_i$ единиц, для всех правильных разбиений это будет выполнятся, а значит не повлияет на их количество. Тогда осталось только разложить оставшиеся, по $n$ местам.
	$$\rchoose{n}{k-s}$$
	
\end{enumerate}


\end{document}
