\documentclass[10pt]{article}
\usepackage[russian]{babel}
\usepackage[utf8]{inputenc}
\usepackage{amssymb}
\usepackage{amsmath}
\usepackage{latexsym}
\usepackage{enumitem}
\usepackage[margin=2cm]{geometry}
\usepackage{pseudocode}

\begin{document}

\title{Домашняя работа 10}
\author{Антон Афанасьев}
\maketitle

\begin{enumerate}

\item[10.1] Отсортируем заявки по неубыванию времени начала. Будем распределять их по аудиториям жадным алгоритмом --- каждой заявке присваиваем минимальный номер не занятой сейчас аудитории. Очевидно, что минимальное количество аудиторий $\chi$ не меньше максимального количества пересекающихся заявок $R$. Наш алгоритм находит допустимое распределение по $R$ аудиториям, а значит является оптимальным.

Докажем это. Предположим, что это не так, то есть наш алгоритм найдет распределение в более чем $R$ аудиторий. Тогда, в момент когда мы назначаем заявке аудиторию, для какой-то заявки заняты все $R$ аудиторий и мы назначаем ей аудиторию $R+1$, однако, это значит, что в этой точке пересекаются более чем $R$ интервалов. Получили противоречие, значит мы найдем распределение в $R$ аудиторий.

\item[10.4] Пусть заказы упорядочены по возрастанию номеров. Тогда $e_i = \sum_{j=0}^i t_j$. Функция, которую нужно минимизировать $L = \sum_i e_i$.

Пусть $t_p < t_q$ и $p > q$. Поменяем работы $q$ и $p$ местами. Тогда $e_i$ до $q$ и после $p$ не изменятся, а между ними уменьшатся, так как вместо $t_q$ в них будет $t_p$, которое меньше. Таким образом целевая функция уменьшилась. Если $t_p > t_q$, то поменяв их местами мы только увеличим целевую функцию по тем же соображениям. 

Значит, отсортировав заказы по неубыванию времени выполнения мы получим нужный порядок.

\end{enumerate}
\end{document}
