\documentclass[10pt]{article}
\usepackage[russian]{babel}
\usepackage[utf8]{inputenc}
\usepackage{amssymb}
\usepackage{amsmath}
\usepackage{latexsym}
\usepackage{enumitem}
\usepackage[margin=2cm]{geometry}
\usepackage{tikz}
\usepackage{graphicx}
\usetikzlibrary{arrows}

\renewcommand{\leq}{\leqslant}
\renewcommand{\geq}{\geqslant}

\begin{document}

\title{Домашняя работа 4}
\author{Антон Афанасьев}
\maketitle

\begin{enumerate}
\item[1.4] Возьмем некоторые три горизонтальные прямые и одну вертикальную. Из трех точек пересечения по принципу Дирихле хотя бы две будут одного цвета. Так как возможных вариантов раскраски этих точек 8, а вертикальных прямых бесконечное количество, то найдутся две вертикальные прямые, для которых цвета точек пересечения с горизонтальными совпадают. Для таких вертикальных прямых выберем те две из трех горизонтальных, на которых лежат точки одного цвета. Получим две горизонтальных и две вертикальных прямые, на пересечении которых лежат точки одного цвета.

\item[5.5] Пусть плоскость покрашена в цвета 1, 2, 3. Предположим, что единичного отрезка, у которого концы покрашены в один цвет нет. 

Возьмем точку $x$, пусть она покрашена в цвет 1. Тогда вокруг этой точки на окружности радиуса один все точки покрашены в цвета 2 и 3. Рассмотрим правильный треугольник со стороной 1. Поставим его основанием на окружность, так чтобы третья вершина была вне круга. У треугольника все вершины окрашены в разные цвета. Так как вершины на окружности покрашены в цвета 2 и 3, третья вершина окрашена в цвет 1. Будем перемещать этот треугольник по окружности, тогда его вершина опишет вокруг точки $x$ окружность радиуса $\sqrt{3}$, причем все точки этой окружности цвета 1. Выберем на этой окружности две точки на расстоянии один, они будут одного цвета. Получили противоречие.

\end{enumerate}

\end{document}
