\documentclass[10pt]{article}
\usepackage[russian]{babel}
\usepackage[utf8]{inputenc}
\usepackage{amssymb}
\usepackage{amsmath}
\usepackage{latexsym}
\usepackage{enumerate}
\usepackage[margin=2cm]{geometry}
\usepackage{tikz}
\usetikzlibrary{arrows}

\begin{document}

\title{Домашняя работа 1}
\author{Антон Афанасьев}
\maketitle

\begin{enumerate}
	\item Покажем по индукции. Для $k=1$ выполняется. Пусть выполняется для $k-1$. Тогда для $k$:\\
	Подсчитаем $b_{i,j}$ - число путей длины $k$ из $i$ в $j$. Для этого переберем предпоследнюю вершину.
	$$b_{i,j} = \sum_{t=1}^n a_{i,t}^{k-1} \cdot a_{t,j}$$ 
	Но тогда $b_{i,j} = a_{i,j}^k$, т.к. $A^k = A^{k-1} \cdot A$.

	\item Пусть вершины $u$ и $v$ связанные, это значит что существует маршрут $u \leadsto v$. Пусть этот маршрут не является простым путем. Это значит, что есть вершина, которая повторяется в маршруте. Возьмем отрезок маршрута между двумя повторяющимися вершинами и выкинем его, связность от этого не пострадает:
	$u \leadsto t \leadsto t \leadsto v \Rightarrow u \leadsto t \leadsto v$. Продолжим делать так пока не останется простой путь.\\
	То есть, если существует маршрут, то существует и простой путь, а каждый простой путь является маршрутом. Значит можно использовать понятие простого пути в определении связанности двух вершин.
	
	\item Рассмотрим компоненты связанности графа $G$. Пусть вершины с нечетной степенью лежат в разных компонентах. Это значит, что в этих компонентах есть ровно одна вершина нечетной степени, но такого быть не может, т.к. сумма степеней вершин в графе четна (а компонента тоже граф). Следовательно, вершины лежат в одной компоненте связанности.
	
	\item Рассмотрим граф \\
\begin{tikzpicture}[auto,main node/.style={circle,fill=blue!20,draw}]
  \node[main node] (1) {1};
  \node[main node] (2) [right of=1] {2};
  \node[main node] (3) [right of=2] {3};

  \path[every node/.style={font=\sffamily\small}]
    (1) edge node [left] {} (2)
    (2) edge node [left] {} (3);
\end{tikzpicture}\\
	Маршруты из 1 в 3
	\begin{itemize}
		\item 1, 2, 3, 2, 3
		\item 1, 2, 3
	\end{itemize}
	Не содержат простой цикл.
	
	\item Возьмем вершину $v$ после которой два пути ``разойдутся'' (такая вершина всегда есть, так как пути различны). Рассмотрим первую после $v$ вершину $u$, общую для обоих путей (такая вершина тоже всегда есть, так как пути должны сойтись по крайней мере в конечной вершине). Цикл из двух половинок $v \leadsto u$ по одному пути и $u \leadsto v$ по другому будет простым циклом.
	
	\item В графе без циклов максимум $m=n-1$ ребер. Каждое следующее ребро будет замыкать какой-либо путь, давая по крайней мере  еще один цикл. А значит циклов минимум $m-n+1$.
	
	\item Для того чтобы гарантированно быть связным граф на $n$ вершинах должен иметь $\frac{(n-1)(n-2)}{2} + 1$ ребро, так как чтобы не дать никакой вершине быть не связанной потребуется полностью насытить ребрами все остальные (в худшем случае будет полный граф на $n-1$ вершине).
	
	\item Введем дополнительную, $n+1$ вершину. Будем считать, что к ней подвешены все корни деревьев. Таким образом мы получили одно дерево, причем каждому лесу соответствует такое дерево и наоборот. Число деревьев на $n+1$ вершине по формуле Кэли равно $(n+1)^{n-1}$, а значит число корневых лесов на $n$ вершинах тоже равно $(n+1)^{n-1}$.
	
	\item Докажем от противного. Пусть $P_1=(x_0,x_1,\ldots,x_k)$ и $P_2=(y_0, y_1, \ldots, y_k)$ --- два максимальных простых пути, не содержащие общей вершины. Возьмем две вершины $u \in P_1$ и $v \in P_2$ достижимые друг из друга по пути, промежуточные вершины которого (если они есть) не лежат в $P_1$ и $P_2$ (такие вершины всегда найдутся, так как граф связный, а пути не совпадают). Вершины $u$ и $v$ делят $P_1$ и $P_2$ на две части. Возьмем большие из этих частей и соединим путем, соединяющим $u$ и $v$. Пусть части получились размеров $m_1 \ge \frac{k}{2}$ и $m_2 \ge \frac{k}{2}$, и путь между $u$ и $v$ длины $L \ge 1$. Тогда мы получили путь размера $m_1 + m_2 + L > k$. Получили противоречие, что $P_1$ и $P_2$ --- максимальные пути.\\
	Следовательно, в связном графе два максимальных простых пути имеют общую вершину.
\end{enumerate}

\end{document}
