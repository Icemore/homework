\documentclass[10pt]{article}
\usepackage[russian]{babel}
\usepackage[utf8]{inputenc}
\usepackage{amssymb}
\usepackage{amsmath}
\usepackage{latexsym}
\usepackage{enumitem}
\usepackage[margin=2cm]{geometry}


\newcommand{\rchoose}[2]{\left(\mkern-6mu \left({#1 \atop #2}\right) \mkern-6mu \right)}

  
\begin{document}

\title{Домашняя работа 1}
\author{Антон Афанасьев}
\maketitle

\begin{enumerate}[label*=1.\arabic*]
	\item $5*4=20$
	\item Рассмотрим числа ровно из $n$ знаков. На первое место можно поставить 9 цифр (лидирующий 0 нельзя, иначе в числе будет меньше $n$ знаков). На второе место можно ставить все цифры, кроме той, что на первом месте, тоесть 9. На $i$-ое место можно ставить все цифры, кроме той, что на $i-1$ месте. Всего таких чисел $9^n$. Соответственно нужных чисел от $0$ до $10^n$ $$ \sum_{i=0}^n 9^i = \frac{1}{8} (9^{n+1} -1) $$
	Ответ: $\frac{1}{8} (9^{6+1} - 1) = 597871$
	
	\item В трехзначном числе (допуская лидирующие нули) поставить цифру 7 -- 3 варианта, и выбрать остальные цифры $9^2$ вариантов. \\
	Ответ: $3*9^2 = 243$
	
	\item Посчитаем через дополнение. Всего чисел от $0$ до $999$ -- $10^3$. Чисел, в которых нет ни одной 7 -- $9^3$ (выбрать каждую цифру не семерку). \\
	Ответ: $10^3 - 9^3 = 271$
	
	\item $D$ -- множество чисел от 1 до 999\\
	$A$ -- множество чисел из $D$, которые делятся на 2\\
	$B$ -- множество чисел из $D$, которые делятся на 3 \\
	$C$ -- множество чисел из $D$, которые делятся на 5 \\
	$|A| = 499, |B| = 333, |C| = 199$ \\
	$|A \cap B| = 166, |A \cap C| = 99, |B \cap C| = 66$ \\
	$|A \cap B \cap C| = 33$\\
	По формуле включений-исключений:\\
	$|A' \cap B' \cap C'| = |D| - |A| -|B|-|C| + |A \cap B|+|A \cap C| +|B \cap C|-|A \cap B \cap C|$ \\
	Ответ: 266
	
	\item Обобщенное правило суммы:\\
	$|A \cup B \cup C| = |A| + |B| + |C| - |A \cap B| - |A \cap C| - |B \cap C| + |A \cap B \cap C|$\\
	$|A \cap B \cap C| = |A \cup B \cup C|-|A| - |B| - |C| + |A \cap B| + |A \cap C| + |B \cap C|$\\
	Так как $|A \cap B| = |A| + |B| - |A \cup B|$ \\
	$|A \cap B \cap C| = |A \cup B \cup C|-|A| - |B| - |C| + |A| + |B| - |A \cup B| + |A| + |C| - |A \cup C| + |B| + |C| - |B \cup C|$\\
	$|A \cap B \cap C| = |A| + |B| + |C| - |A \cup B|- |A \cup C|- |B \cup C| + |A \cup B \cup C|$
\end{enumerate}

\begin{enumerate}[label*=2.\arabic*]
	\item Рассмотрим множество $n$ элементных подмножеств $(m+n+1)$-элементного множества. Возьмем все подмножества не содержащие элемент $x_{m+n+1}$, их ${m+n \choose n}$. Возьмем все подмножества содержащие $x_{m+n+1}$, но не содержащие $x_{m+n}$, их ${m+n-1 \choose n-1}$. Возьмем содержащие $x_{m+n+1}$ и $x_{m+n}$, но не содержащие $x_{m+n-1}$, их ${m+n-2 \choose n-2}$. Продолжая выбирать подмножества таким образом получим разбиение исходного множества, а значит $$\sum_{k=0}^n {m+k \choose k} = {m+n+1 \choose n}.$$
	
	\item Пусть есть множества $A, B, |A|=n, |B|=m$. Хотим посчитать ${n+m \choose k}$. Переберем сколько элементов будем брать из $A$. Пусть из $A$ берем $i$, тогда из $B$ нужно взять $k-i$. Число способов это сделать ${n \choose i} \cdot {m \choose k-i}$. Следовательно $${n+m \choose k} = \sum_{i=0}^k {n \choose i} \cdot {m \choose k-i}.$$
	
	\item Хотим посчитать $\rchoose{n}{k}$. Рассмотрим подмножества которые не содержат элемент $x_n$, их $\rchoose{n-1}{k}$. Рассмотрим подмножества, которые содержат хотя бы один элемент $x_n$, их $\rchoose{n}{k-1}$ (одну позицию занимает элемент $x_n$, туда не нужно ничего выбирать). Эти множества образуют разбиение исходного множества, а значит $$\rchoose{n}{k} = \rchoose{n-1}{k} + \rchoose{n}{k-1}.$$
	
	\item Хотим посчитать $\rchoose{n+1}{k}$. Переберем число вхождений элемента $x_{n+1}$ в подмножества. Число подмножеств, в которые $x_{n+1}$ входит $i$ раз равно $\rchoose{n}{k-i}$. Эти множества образуют разбиение исходного множества, а значит $$\rchoose{n+1}{k} = \sum_{i=0}^k \rchoose{n}{k-i}.$$
	
	$${n+k \choose n+1} = {n+k \choose k-1} = \rchoose{n+2}{k-1} = \sum_{i=0}^{k-1} \rchoose{n+1}{k-1-i} = \sum_{i=0}^{k-1} \rchoose{n+1}{i} = \sum_{i=0}^{k-1} \binom{n+i}{i} =$$  $$\sum_{i=0}^{k-1} \binom{n+i}{n} = \sum_{i=1}^{k} \binom{n+i-1}{n} = \sum_{i=0}^{k} \binom{n+i-1}{n} = \sum_{i=1}^{k} \binom{n+k - i-1}{n}.$$
	
	\item Суммы
	\begin{enumerate}
	\item $$ \sum_{i=0}^n i = \sum_{i=0}^n \binom{i}{1} = \binom{n+1}{2} = \frac{(n+1)n}{2}$$
	\item $$\binom{n+1}{3} = \sum_{i=0}^n \binom{i}{2} = \sum_{i=0}^n \frac{i(i-1)}{2} = \frac{1}{2} \left[ \sum_{i=0}^n {i^2} - \sum_{i=0}^n {i} \right] $$
	
	$$\Rightarrow \sum_{i=0}^n {i^2} = 2 \cdot \binom{n+1}{3} + \binom{n+1}{2}$$
	
	\item $$\binom{n+1}{4} = \sum_{i=0}^n \binom{i}{3} = \sum_{i=0}^n \frac{i(i-1)(i-2)}{2 \cdot 3} = \frac{1}{6} \left[ \sum_{i=0}^n {i^3} - 3 \sum_{i=0}^n {i^2} + 2 \sum_{i=0}^n {i} \right]$$
	
	$$\Rightarrow \sum_{i=0}^n {i^3} = 6 \cdot \binom{n+1}{4} + 6 \cdot \binom{n+1}{3} + \binom{n+1}{2} $$
	\end{enumerate}
\end{enumerate}

\end{document}
