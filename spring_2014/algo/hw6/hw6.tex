\documentclass[10pt]{article}
\usepackage[russian]{babel}
\usepackage[utf8]{inputenc}
\usepackage{amssymb}
\usepackage{amsmath}
\usepackage{latexsym}
\usepackage{enumitem}
\usepackage[margin=2cm]{geometry}
\usepackage{relsize}

\newcommand{\rchoose}[2]{\left(\mkern-6mu \left({#1 \atop #2}\right) \mkern-6mu \right)}
\newcommand{\dsum}{\sum_{\substack{ k\le n \\ k - \text{нечетное}}}}
\renewcommand{\P}{\text{Pr}}

\begin{document}

\title{Домашняя работа 6}
\author{Антон Афанасьев}
\maketitle

\begin{enumerate}
	\item[(a)] Так как хэш присваивается ключам независимо, то для $n$ ключей у нас есть $\binom{n}{k}$ способов выбрать каким ключам будет присвоено значение $x$. $k$ этих ключей примут значение $x$ с вероятностью $\left ( \frac{1}{n} \right ) ^k$, а остальные примут какое-то другое значение с вероятностью $\left ( 1- \frac{1}{n} \right ) ^ {n-k} $.
	
	Итого, $$Q_k = \binom{n}{k} \cdot \left ( \frac{1}{n} \right ) ^k \cdot \left ( 1- \frac{1}{n} \right ) ^ {n-k} $$
	
	\item[(b)] Чтобы самая длинная цепочка была длины $k$, в какой-то ячейке она должна быть такой длины, а во все остальных --- меньше. Тогда, $P_k = \sum_{x=1}^n Q_k \cdot T_k$, где $T_k$ --- вероятность, что в остальных ячейках цепочки меньшей длины.
		$$P_k \le \sum_{x=1}^n Q_k \le n\cdot Q_k$$
\end{enumerate}
\end{document}
