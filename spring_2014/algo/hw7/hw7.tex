\documentclass[10pt]{article}
\usepackage[russian]{babel}
\usepackage[utf8]{inputenc}
\usepackage{amssymb}
\usepackage{amsmath}
\usepackage{latexsym}
\usepackage{enumitem}
\usepackage[margin=2cm]{geometry}
\usepackage{relsize}

\newcommand{\rchoose}[2]{\left(\mkern-6mu \left({#1 \atop #2}\right) \mkern-6mu \right)}
\newcommand{\dsum}{\sum_{\substack{ k\le n \\ k - \text{нечетное}}}}
\renewcommand{\P}{\text{Pr}}

\begin{document}

\title{Домашняя работа 7}
\author{Антон Афанасьев}
\maketitle

\begin{enumerate}
	\item[1.a] Рассмотрим числа как произведение простых в некоторых степенях $a = p_1 ^{a_1} \cdot p_2^{a_2} \cdot p_3^{a_3} \ldots$, $b = p_1^{b_1} \cdot p_2^{b_2} \cdot p_3^{b_3} \ldots$. Тогда НОД этих числе будет равен числу с максимальной степенью каждого простого $gcd(a, b) = p_1^{\max(a_1, b_1)} \cdot p_2^{\max(a_2, b_2)} \cdot p_3^{\max(a_3, b_3)} \ldots$.
	
	В таком представлении очевидно, что если $2 | a$ и $2 | b$, то $gcd(a, b) = 2 \cdot gcd(a/2, b/2)$. Если же 2 не входит в одно из чисел $a$ и $b$, то оно не войдет и в $gcd(a, b)$ и можно его убрать из рассмотрения. Таким образом в алгоритме сначала из чисел убираются двойки (и пока возможно они увеличивают gcd), а потом запускается обычный алгоритм Евклида.
	
	\item[2] Будем перебирать числа от 2 до $\sqrt[3]{n}$, обозначим перебираемое число как $t$. Для каждого $t$ будем проверять можно ли представить $n = t^2 \cdot y$ (дважды разделив нацело $n$ на $t$). Если два раза раза разделить нельзя будем делить один раз, уменьшая $n$. Пусть мы не нашли квадрата и у нас осталось число $k$. Если квадрат в нем всетаки есть, и $k = x^2 \cdot y$, то $x > \sqrt[3]{n}$, так как все остальные мы уже проверили. Тогда, $x^2 > n^{\frac{2}{3}}$ и $y < n^{\frac{1}{3}}$, но на все числа мешьние $\sqrt[3]{n}$ мы уже поделили, значит, единственным смособом содержать квадрат для $k$ остается быть полным квадратом $k = x^2$. А это можно проверить, напрмер бинарным поиском за $O(\log n)$.
	
	Получили трудоемкость $O(n^{\frac{1}{3}} + \log n) = O(n^{\frac{1}{3}})$
	\end{enumerate}
\end{document}
