\documentclass[10pt]{article}
\usepackage[russian]{babel}
\usepackage[utf8]{inputenc}
\usepackage{amssymb}
\usepackage{amsmath}
\usepackage{latexsym}
\usepackage{enumitem}
\usepackage[margin=2cm]{geometry}
\usepackage{relsize}

\newcommand{\rchoose}[2]{\left(\mkern-6mu \left({#1 \atop #2}\right) \mkern-6mu \right)}
\renewcommand{\P}[1]{\left ( 1 - \frac{1}{m} \right ) ^ {k \cdot #1} }
\renewcommand{\not}{\overline}

\begin{document}

\title{Контрольная задача}
\author{Антон Афанасьев}
\maketitle

\begin{enumerate}
\item[1.]
	Пусть у нас фильтры Блума с таблицей размера $m$ и с $k$ хэш-функциями. Будем считать, что при добавлении элемента каждая хэш-функция ставит 1 в ячейке с вероятностью $\frac{1}{m}$. Тогда, после $n$ добавлений, вероятность, что в ячейке будет стоять 0: $\P{n} = \not{P_n}$, а вероятность, что в ячейке будет стоять 1: $P_n = 1 - \not{P_n}$.
	
	Найдем вероятность 1 в ячейке пересечения фильтров Блума. 1 может стоять в ячейке благодаря тому, что она поставлена элементом из пересечения множеств. Если это не так, то 1 стоит потому, она была поставлена в одном и в другом фильтре элементами не входящими в пересечение. Таким образом, вероятность 1:
	$$P_c + (1 - P_c) \cdot P_{n_1 - c} \cdot P_{n_2 - c}=$$
	$$1 - \P{c} + \P{c} \cdot \left [ 1 - \P{(n_1-c)} \right ] \cdot \left [ 1 - \P{(n_2-c)} \right ] =$$
	$$1 + \P{c} \cdot \left [ -1 + 1 - \P{(n_1 - c)} - \P{(n_2-c)} + \P{(n_1 + n_2 - 2c)} \right ] = $$
	$$1 - \P{n_1} - \P{n_2} + \P{(n_1 + n_2 - c)}$$
	
	Ложное срабатывание происходит тогда, когда во всех $k$ ячейках для запрашиваемого элемента оказались 1. Следовательно, вероятность ложного срабатывания:
	$$\left [ 1 - \P{n_1} - \P{n_2} + \P{(n_1 + n_2 - c)} \right ] ^k$$
\end{enumerate}
\end{document}
