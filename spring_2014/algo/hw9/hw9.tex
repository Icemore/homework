\documentclass[10pt]{article}
\usepackage[russian]{babel}
\usepackage[utf8]{inputenc}
\usepackage{amssymb}
\usepackage{amsmath}
\usepackage{latexsym}
\usepackage{enumitem}
\usepackage[margin=2cm]{geometry}
\usepackage{relsize}

\newcommand{\rchoose}[2]{\left(\mkern-6mu \left({#1 \atop #2}\right) \mkern-6mu \right)}
\newcommand{\dsum}{\sum_{\substack{ k\le n \\ k - \text{нечетное}}}}
\renewcommand{\P}{\text{Pr}}

\begin{document}

\title{Домашняя работа 9}
\author{Антон Афанасьев}
\maketitle

\begin{enumerate}
	\item[1.] Возьмем два полинома степени $10 \cdot n$: a и b. В полиномах при степени $i$ поставим коэффициент 1, если число $i$ есть в соответствующем множестве и 0 если нет. Перемножим эти полиномы и получим полином c. Т.к. $c_n = \sum_{i=0}^n a_i \cdot b_{n-i}$, то $c_n \neq 0$, только если есть такое $i$, что $a_i \neq 0$ и $b_{n-i} \neq 0$. А это значит, что в множестве $A$ есть число $i$, в множестве $B$ есть число $n-i$, следовательно, в множестве $C$ есть число $n$. Таким образом, множеству $C$ принадлежат все $n$, такие что $c_n \neq 0$.
	
	\item[2.] Повторим каждое $z_i$ $\alpha_i$ раз, чтобы получить $n$ скобок в первой степени. Будем решать задачу рекурсивно --- рекурсивно запустимся от первой и второй половины скобок, а потом перемножим результат за $O(n \log n)$.
	
	Уравнение на время работы: $T(n) = 2 \cdot T(n/2) + n \log n$. Несложно видеть, что его можно оценить как $O(n \log^2 n)$. Для этого представим себе дерево вызовов, на каждом уровне $n$ уменьшается вдвое. На уровне с $k$ вершинами в каждой вершине выполняется $\frac{n}{k} \log \frac{n}{k}$ операций. Всего на уровне выполняется $n \log \frac{n}{k}$ операций, что меньше, чем $n \log n$. Всего уровней $\log n$, значит $T(n) = O(n \log^2 n)$.
	
	\item[3.] Будем хранить матрицу вектором элементов --- первый столбец снизу вверх, потом первая строка слева направо. Строка $i$ матрицы совпадает с элеметами $m$ элементами вектора, начиная с $n-i$. Для умножения матрицы $M$ на вектор $x$ умножим вектор представляющий $M$ на перевернутый вектор $x$ при помощи быстрого преобразования Фурье. Идея такая же как для вычисления $\sum a_i \cdot b_i$.  Последние $n$ коэффициентов полученного полинома будут элементами вектора произведения.
\end{enumerate}
\end{document}
