\documentclass[10pt]{article}
\usepackage[russian]{babel}
\usepackage[utf8]{inputenc}
\usepackage{amssymb}
\usepackage{amsmath}
\usepackage{latexsym}
\usepackage{enumitem}
\usepackage[margin=2cm]{geometry}
\usepackage{relsize}

\newcommand{\rchoose}[2]{\left(\mkern-6mu \left({#1 \atop #2}\right) \mkern-6mu \right)}
\newcommand{\dsum}{\sum_{\substack{ k\le n \\ k - \text{нечетное}}}}
\renewcommand{\P}{\text{Pr}}

\begin{document}

\title{Домашняя работа 12}
\author{Антон Афанасьев}
\maketitle

\begin{enumerate}
	\item[1.] Построим суффиксное дерево. Каждая его вершина соответствует подстроке. Заметим, что рассматривать ``буквы'' на ребре нет смысла, так как подстрока соответствующая вершине на конце этого ребра встречается такое же количество раз и при этом длиннее. Количество вхождений подстроки можно подсчитать как количество листьев в этом поддереве. Обходом в глубину обойдем дерево поддерживая глубину (длину подстроки) и считая количество листьев в поддеревьях. В каждой вершине перемножим эти две величины и из них возьмем максимальную.
	
	\item[2.] Построим Z-функцию по строке $s + `\#` + t$. Будем идти по строке с того места, где начинается $t$. Пусть до индекса $j$ мы уже разбили строку на префиксы $s$. Максимальный префикс котрый мы можем использовать, это $z[j]$, он закончится в $r = j + z[j]$. Мы можем закончить текущий префикс в любом месте от $j$ до $r$. Будем двигаться по строке направо. Пусть в какой-то момент $i + z[i] > r$. Тогда закончим текущий префикс (начинающийся в $j$) здесь ($i$) и включим его в разбиение. Так как $i+z[i] >r$, то следующим префиксом можно будет покрыть большую часть строки, и мы не ухудшили положение (раньше мы могли покрыть строку только до $r$, а теперь до $i+z[i]$) и строки начинающиеся до $r$ мы по прежнему рассмотрим. 
	
	\item[3.]  Фактически, от нас требуется сказать для каждого префикса четной длины строки $s$ правда ли, что он является палиндромом. Если это правда, то строка $x$ может быть получена путем добавления оставшейся (не вошедшей в префикс) части строки $s$ в начало с переворотом. 
	
	Склеим строку $s$ с перевернутой $s$ через \#. Построим для такой строки Z-функцию. Посмотрим на значение Z-функции на всех суффиксах перевернутой строки. Каждый такой суффикс соответствует префиксу неперевернутой строки. Для того чтобы этот префикс был палиндромом достаточно, чтобы значение Z-функции на соответствующем суффиксе было больше либо равно половины его длины. Это как раз будет означать, что префикс можно разбить пополам так, что вторая половина равна перевернутой первой.
		
\end{enumerate}
\end{document}
