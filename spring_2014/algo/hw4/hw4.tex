\documentclass[10pt]{article}
\usepackage[russian]{babel}
\usepackage[utf8]{inputenc}
\usepackage{amssymb}
\usepackage{amsmath}
\usepackage{latexsym}
\usepackage{enumitem}
\usepackage[margin=2cm]{geometry}
\usepackage{pseudocode}
\usepackage{hyperref}

\hypersetup{colorlinks=true, urlcolor=blue}

\begin{document}

\title{Домашняя работа 4}
\author{Афанасьев Антон}
\maketitle

\begin{enumerate}
	\item[1.] Заведем дерево отрезков. В вершине будем хранить цвет ячеек на всем этом отрезке либо факт отсутствия одного цвета. Теперь при запросе цвета конкретной ячейки спускаемся до нее по дереву, пока не встретим покрашенный отрезок. Для вывода покраски всего массива обойдем дерево поиском в глубину. Когда встречаем покрашенный отрезок проталкиваем его вниз, если это еще не лист (говорим что оба его ребенка покрашены в этот цвет). Так как вершин в дереве линейное количество, то эта операция займет линейное время.\\
	При покраске отрезка в новый цвет будем спускаться по дереву, и, если текущая вершина имеет какой-то цвет, то перед тем как что-то делать протолкнем этот цвет в ее детей. Спускаемся так же как с обычной операцией на отрезке, когда вершина полностью соответствует отрезку сохраняем в ней цвет.
	
	\item[2.] Заведем декартово дерево по неявному ключу со случайными приоритетами. Будем хранить в нодах значения ячеек и поддерживать на них операцию минимума. \\
	Для изменения значения в $a[i]$ сделаем split дерева выделив эту вершину, изменим в ней значение и смержим все назад. \\
	Для второй операции сделаем split выделив отрезок $(i+1, n)$. Найдем в полученном дереве самую левую вершину меньшую $a[i]$ (если ее нет мы можем сразу это сказать посмотрев на минимум в этом дереве). Для этого будем спускаться по дереву. Находясь в вершине $v$ посмотрим на минимум в левом поддереве и, если он меньше $a[i]$ пойдем туда. Если нет, то сравним $a[i]$ с самой $v$ и, в зависимости от результата, либо вернем $v$, либо пойдем в правое поддерево. Так как высота дерева логарифмическая операция выполняется за $O(\log n)$

\end{enumerate}

\end{document}

