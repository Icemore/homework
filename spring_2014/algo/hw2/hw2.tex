\documentclass[10pt]{article}
\usepackage[russian]{babel}
\usepackage[utf8]{inputenc}
\usepackage{amssymb}
\usepackage{amsmath}
\usepackage{latexsym}
\usepackage{enumitem}
\usepackage[margin=2cm]{geometry}
\usepackage{pseudocode}
\usepackage{hyperref}

\hypersetup{colorlinks=true, urlcolor=blue}

\begin{document}

\title{Домашняя работа 2}
\author{Афанасьев Антон}
\maketitle

\begin{enumerate}
	\item Код \url{http://pastebin.com/ggAxULNe} \\
	Посылка на тимусе номер 5525017.
	
	\item Положим элементы с четными и нечетными номерами в два splay-дерева по неявному ключу. Будем поддерживать в них сумму в поддереве. Теперь для ответа на запрос первого типа нужно вырезать соответствующие отрезки из первого и второго дерева и поменять их местами (четные элементы становятся нечетными и наоборот). Для ответа на запрос второго типа нужно просто взять сумму на отрезке в обоих деревьях (для этого вырезав нужный отрезок и посмотрев сумму в поддереве).
	
	Код \url{http://pastebin.com/CWRQntt7}\\
	Посылка на e-olymp \href{http://www.e-olimp.com/solutions/1361392}{1361392}.
	

\end{enumerate}

\end{document}

