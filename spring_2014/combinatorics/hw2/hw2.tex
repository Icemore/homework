\documentclass[10pt]{article}
\usepackage[russian]{babel}
\usepackage[utf8]{inputenc}
\usepackage{amssymb}
\usepackage{amsmath}
\usepackage{latexsym}
\usepackage{enumitem}
\usepackage[margin=2cm]{geometry}
\usepackage{relsize}

\newcommand{\rchoose}[2]{\left(\mkern-6mu \left({#1 \atop #2}\right) \mkern-6mu \right)}
\newcommand{\dsum}{\sum_{\substack{ k\le n \\ k - \text{нечетное}}}}
\renewcommand{\P}{\text{Pr}}

\begin{document}

\title{Домашняя работа 2}
\author{Антон Афанасьев}
\maketitle

\begin{enumerate}
	\item[4.2.] Покажем, что при $m > 0$ в разбиении всегда есть 1. Рассмотрим диаграмму Ферре для разбиения $2n+m$ на $n+m$ слагаемых. Первый столбец обязательно заполнен (так как разбиение ровно на $n+m$ слагаемых). Если убрать первый столбец в диаграмме останется $n$ элементов. Даже если все их выстроить в один столбец их не хватит, чтобы создать полностью заполненный второй столбец в исходной диаграмме (при ненулевом $m$). Значит, в каждом таком разбиении есть 1. 
	
	Возьмем произвольное разбиение для некоторого $m>0$. Уберем последнюю 1. Получим разбиение для $m-1$. И в обратную сторону, можно приписать 1 к любому разбиению и получить разбиение для $m+1$. Значит, количество разбиений $2n+m$ на ровно $n+m$ одинаково для любого $m$.
	
	Подсчитаем это количество при $m=0$. Оно равно $p_n(2n)$. Пользуясь утверждением 4.2:
	$$p_n(2n) = \sum_{r=1}^n p_r(n) = p(n)$$
	
	\item[4.6.] Рассмотрим следующую диаграмму Ферре: возьмем квадрат из $n^2$ точек со стороной $n$. Возьмем две произвольные диаграммы Ферре, из $n$ точек каждая, и приставим их к квадрату одну справа, другую снизу. Получим допустимую диаграмму Ферре из $n^2 + 2n$ точек, соответствующую какому-то разбиению числа $n^2 + 2n$. Получили, что любой паре разбиений числа $n$ можем однозначно сопоставить разбиение $n^2 + 2n$, причем отображение инъективное. Значит, $p(n)^2 < p(n^2 + 2n)$ (неравенство строгое, так как очевидно, что таким способом нельзя получить все разбиения, например разбиение на все единицы будет отсутствовать).
	
	\item[4.8.] Рассмотрим диаграмму Ферре для разбиения в котором равны три наибольшие части. Повернем диаграмму на $90^\circ$. Получим диаграмму в которой минимальное число как минимум 3. Подсчитаем количество таких диаграмм. Для этого подсчитаем количество диаграмм, в которых минимальный элемент 1 или 2. Минимальный элемент 1 --- просто количество диаграмм на $n-1$ элементах (можно убрать последнюю 1 для перехода к $n-1$, и приписать для перехода назад), оно равно $p(n-1)$. Минимальный элемент 2 --- количество диаграмм на $n-2$ элементов, которые не заканчиваются на 1 (отображение --- приписать и убрать 2), таких $p(n-2) - p(n-3)$.
	
	Итого, ответ: $p(n) - p(n-1) - p(n-2) + p(n-3)$.
	
	\item[4.9.] Возьмем следующую диаграмму Ферре: 

\setlength{\unitlength}{1mm}
\begin{picture}(15, 12)
\put(15,10){\circle*{1}} \put(20,10){\circle*{1}} \put(25,10){\circle*{1}} \put(30,10){\circle*{1}}
\put(15,5){\circle*{1}} \put(20,5){\circle*{1}} \put(25,5){\circle*{1}}
\put(15,0){\circle*{1}} \put(20,0){\circle*{1}}
\put(15,-5){\circle*{1}}
\end{picture}
	
	\vspace{0.5cm}
	
	``Лесенка'' $k-1$ на $k-1$ из $k(k-1)/2$ точек. Заметим, что если взять любое разбиение числа $n + k(k-1)/2$ на $k$ различных слагаемых, то в соответствующей диаграмме Ферре такая лесенка обязательно будет содержаться (прижатая к левому верхнему углу). Это легко показать поднимаясь по диаграмме снизу вверх. В последней строке как минимум 1 точка, в предпоследней как минимум 2 (так как числа различны) и т.д. Причем, если удалить лесенку из диаграммы останется правильная диаграмма Ферре, описывающая разбиения числа $n$ на $k$ слагаемых.
	 
	Приписывая лесенку к произвольному разбиению $n$ на $k$ слагаемых получаем разбиение на различные, так как к большим числам прибавляются большие (неравные числа никогда не станут равны), а к равным прибавляются разные значения.
	
	Так как такая процедура проделывается однозначно, имеем биекцию между разбиениями $n$ на $k$ слагаемых и разбиениями $n + k(k-1)/2$ на $k$ различных слагаемых, а значит и количество таких разбиений совпадает.

\end{enumerate}
\end{document}
