\documentclass[10pt]{article}
\usepackage[russian]{babel}
\usepackage[utf8]{inputenc}
\usepackage{amssymb}
\usepackage{amsmath}
\usepackage{latexsym}
\usepackage{enumitem}
\usepackage[margin=2cm]{geometry}
\usepackage{relsize}

\newcommand{\rchoose}[2]{\left(\mkern-6mu \left({#1 \atop #2}\right) \mkern-6mu \right)}
\newcommand{\dsum}{\sum_{\substack{ k\le n \\ k - \text{нечетное}}}}
\renewcommand{\P}{\text{Pr}}

\begin{document}

\title{Домашняя работа 1}
\author{Антон Афанасьев}
\maketitle

\begin{enumerate}
\item[1.1.] Для того, чтобы доказать, что $D_n$ ближайшее к $n!e^{-1}$ целое число, покажем, что $|D_n - n!e^{-1}| < 0.5$.
$$D_n = \sum_{i=0}^n (-1)^i \binom{n}{i} (n-i)! = n! \sum_{i=0}^n (-1)^i \frac{1}{i!}$$
$$e^{-1} = 1 - \frac{1}{1!} + \frac{1}{2!} - \frac{1}{3!} + \ldots = \sum_{i=0}^\infty (-1)^i \frac{1}{i!}$$
$$\left | D_n - n!e^{-1} \right | < 0.5 \Leftrightarrow 
\left |n! \left ( \sum_{i=0}^n (-1)^i \frac{1}{i!} - e^{-1} \right ) \right | < 0.5 \Leftrightarrow
\left |\sum_{i=0}^n (-1)^i \frac{1}{i!} - e^{-1} \right | < \frac{1}{2} \cdot \frac{1}{n!} \Leftrightarrow
\left |\sum_{i=n+1}^\infty (-1)^i \frac{1}{i!} \right | < \frac{1}{2} \cdot \frac{1}{n!}
$$
Пусть $\gamma_n = \sum_{i=n+1}^\infty (-1)^i \frac{1}{i!}$. Это остаток знакопеременного ряда. Из матанализа известно, что остаток знакопеременного ряда меньше его первого первого члена (по модулю). Значит, $|\gamma_n| < \frac{1}{(n+1)!} \le \frac{1}{2 \cdot n!}$ при $n \ge 1$. 

\item[1.2.] $D_{n+1} = n ( D_n + D_{n-1})\ \forall n > 1;\ D_0 = 1, D_1 = 0$

Рассмотрим перестановки из $n+1$ элемента. Посмотрим сколько из них нам подойдет для $D_{n+1}$. На первом месте может стоять $n$ элементов. Пусть на первом месте стоит элемент $x$. Тогда в любой правильной перестановке можно поменять $x$ с 1, вычесть из всех элементов 1, и тогда на позициях $2 \ldots n+1$ получится либо беспорядок из $n$ элементов, либо перестановка $n$ элементов, в которой $x$ стоит на своем месте, а все остальные элементы нет (так как одновременно с вычитанием 1 из всех элементов мы сдвинули их на 1 влево, удалив первый).

Теперь, число беспорядков на $n$ элементов равно $D_n$, а число перестановок, где только $x$ находится на своем месте совпадает с числом беспорядков на $n-1$ элементах --- $D_{n-1}$. Для доказательства последнего утверждения рассмотрим перестановку на $n$ элементах, в которой только $x$ находится на своем месте. Удалим $x$ и уменьшим все элементы большие него на 1. Получим перестановку на $n$ элементах. Покажем, что ни один элемент в этой перестановке не находится на своем месте. Рассмотрим элемент $t$ (до вычитания 1). Пусть $t<x$, тогда для него что-то изменится после удаления, только если он находится правее $x$. Он может сдвинуться на 1 влево, однако место с номером $t$ находится левее $x$ и элемент до него не дойдет. Пусть теперь $t>x$ и он уменьшился на 1 после удаления. Тогда, если он находился левее $x$ он не смог встать на свое место, так как оно больше или равно $x$. Если он находился левее, то для него одновременно уменьшилось значение и произошел сдвиг налево, а значит элемент не может встать на свое место, иначе он был бы на нем до удаления. Следовательно, после удаления $x$ получается беспорядок на $n-1$ элементах. Обратную процедуру (добавления $x$ в беспорядок на $n-1$ элементах) так же можно провести единственным способом, а значит, количество перестановок где только $x$ находится на своем месте равно $D_{n-1}$.

Итак, $D_{n+1} = n(D_n + D_{n-1})$.


\item[1.3.] Пусть мы решили, что во второй и четвертый ящик положим конкретные $k$ предметов. Так как в один можно класть только четное число предметов, а в другой нечетное, то $k$ --- нечетное. Для любой раскладки $k$ предметов по этим ящиком в одном окажется четное число, а в другом нечетное число предметов, при этом количество способов разложить их так, чтобы в первом ящике было четное число предметов равно количеству способов, что в первом ящике нечетное число предметов (в силу того, что один способов получается из другого перестановкой ящиков). Значит, всего способов правильно разложить $k$ предметов в эти ящики $2^k / 2 = 2^{k-1}$.

Подсчитаем теперь ответ на задачу --- просуммируем по всем способам выбрать предметы, которые положим во второй и четвертый ящик
$$ \dsum \binom{n}{k} \cdot 2^{n-k} \cdot 2^{k-1} = \dsum \binom{n}{k} \cdot 2^{n-1} = 2^{n-1} \dsum \binom{n}{k} = 4^{n-1}$$
Так как сумма биномиальных коэффициентов по четному нижнему индексу равна сумме по нечетному нижнему индексу и равна $2^{n-1}$

\item[1.4.] Рассмотрим перегородки между элементами (днями). Выберем 4 из них. После первой поедем в командировку, вторая будет разделителем между лекциями и практиками, перед последними двумя возьмем по отгулу. Каждая расстановка перегородок соответствует выбору расписания, и каждое расписание соответствует выбору перегородок. Значит, ответ $\binom{n+1}{4}$.

\item[1.5.] Производящая функция для комбинаторного действия ``выбрать одного солдата из $n$'': 
$$g(z) = \sum_{n=0}^\infty n z^n = \frac{z}{(1-z)^2}$$
ее можно получить например считая, что мы разбиваем $n$ солдат на 3 группы, первую и последнюю произвольного размера, а вторая размера 1: $\frac{1}{1-z} \cdot z \cdot \frac{1}{1-z}$

Нам необходимо разбить шеренгу на произвольное число интервалов и в каждом совершить комбинаторное действие, описываемое $g(z)$
$$f(z) = \frac{1}{1-g(z)} =\frac{1}{1-\frac{z}{(1-z)^2}} = \frac{(1-z)^2}{(1-z)^2 - z}$$

\item[1.6.] Будем считать, что выбор элементов --- это разбиение отрезка чисел на интервалы длины не менее 4, первое число в начале каждого интервала --- выбранное. Плюс интервал производной длины (возможно нулевой) в начале, в котором не выбрано ни одно число (для того чтобы можно было в качестве первого числа брать не только первое). 

Производящая функция для интервала длины не менее 4: 
$$g_1(z) = \frac{z^4}{1-z}$$
(сначала выбираем интервал размера ровно 4, а за ним произвольный).\\
Производящая функция для действия ``разбить на произвольное число интервалов длины не менее 4'':
$$g_2(z) = \frac{1}{1-g_1(z)} = \frac{1}{1 - \frac{x^4}{1-z}} = \frac{1-z}{1-z-z^4}$$
Значит, ответ: 
$$f(z) = \frac{1}{1-z} \cdot g_2(z) = \frac{1}{1-z-z^4}$$
\end{enumerate}
\end{document}
