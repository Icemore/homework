\documentclass[10pt]{article}
\usepackage[russian]{babel}
\usepackage[utf8]{inputenc}
\usepackage{amssymb}
\usepackage{amsmath}
\usepackage{latexsym}
\usepackage{enumitem}
\usepackage[margin=2cm]{geometry}
\usepackage{relsize}

\newcommand{\rchoose}[2]{\left(\mkern-6mu \left({#1 \atop #2}\right) \mkern-6mu \right)}
\newcommand{\dsum}{\sum_{\substack{ k\le n \\ k - \text{нечетное}}}}
\renewcommand{\P}{\text{Pr}}

\begin{document}

\title{Домашняя работа 3}
\author{Антон Афанасьев}
\maketitle

\begin{enumerate}
	\item[5.2] Производящее функция для действия ``взять блок размера хотя бы два, и ничего с ним больше не делать'' $F(z) = e^z - z - 1$.
	
	Разбиваем на произвольное число блоков и в каждом делаем действие $F(z)$: $H(z) = e^{e^z -z -1}$. Производящая функция чисел Белла $B(z) = e^{e^z - 1}$.\\
	$H(z) = e^{e^z -z -1} = e^{e^z - 1} \cdot e^{-z} = B(z) \cdot e^{-z}$\\
	$e^{-z} = \sum_{n=0}^{\infty} (-1)^n \frac{z^n}{n!}$
	
	Перемножая экспоненциальные функции получим коэффициенты для $H(z)$
	
	$$c_n = \sum_{i=0}^n \binom{n}{i} \cdot B_i \cdot(-1)^{n-i}$$
	
	\item[5.3] Будем считать, что мы делаем следующие действия --- разбить $n$ различимых элементов на  неразличимые блоки, в каждом блоке четного размера $i$ упорядочить все $i!$ способами, а затем упорядочить все блоки факториалом способов.
	
	Производящая функция для действий внутри блоков:
	$$F(z) = z^2 + z^4 + z^6 + \ldots = \frac{z^2}{1 - z^2}$$
	Производящая функция для действий над блоками:
	$$G(z) = 1 + z^1 + z^2 + \ldots = \frac{1}{1-z}$$
	$$H(z) = G(F(z)) = \frac{1}{1 - \frac{z^2}{1-z^2}} = \frac{1 - z^2}{1 - 2 z^2} = (1-z^2) \frac{1}{1-2z^2} = (1-z^2) \sum_{n=0}^{\infty} \left ( 2 z^2 \right ) ^ n = (1-z^2) \sum_{n=0}^{\infty} 2^n z^{2n}$$
	$$H(z) = \sum_{n=0}^{\infty} 2^n z^{2n}  - \sum_{n=0}^{\infty} 2^n z^{2n + 2}$$
	
	Коэффициент при $z^n$, если $n$ четное: $2^{n/2} - 2^{(n-2)/2} = 2^{n/2 - 1}$.
	
	Таким образом, коэффициенты $H(z)$, с учетом того, что это экспоненциальная производящая функция:
	$$
	c_n = 
		\begin{cases}
			n! \cdot 2^{n/2 - 1},&\text{если n четное}\\
			0,&\text{если n нечетное}
		\end{cases}
	$$
	
	\item[6.3] Каждое слагаемое соответствует количеству разбиений $n$-элементного множества на такие блоки, что число блоков размера $i$ равно $k_i$, $1 \le i \le n$. Для каждого слагаемого берем все возможные перестановки ($n!$) из $n$, и говорим, что сначала например идут $k_1$ блоков размера 1, потом $k_2$ блоков размера 2 и так далее. Однако мы подсчитали одинаковые разбиения много раз. В знаменателе каждый множитель вида $i!$ убирает упорядоченность внутри одного блока размера $i$. Так как таких блоков $k_i$, то получается множитель $(i!)^{k_i}$. Каждый множитель вида $k_i!$ убирает упорядоченность блоков одного размера.
\end{enumerate}
\end{document}
