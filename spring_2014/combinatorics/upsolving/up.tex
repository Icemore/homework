\documentclass[10pt]{article}
\usepackage[russian]{babel}
\usepackage[utf8]{inputenc}
\usepackage{amssymb}
\usepackage{amsmath}
\usepackage{latexsym}
\usepackage{enumitem}
\usepackage[margin=2cm]{geometry}
\usepackage{relsize}

\newcommand{\rchoose}[2]{\left(\mkern-6mu \left({#1 \atop #2}\right) \mkern-6mu \right)}
\newcommand{\dsum}{\sum_{\substack{ k\le n \\ k - \text{нечетное}}}}
\renewcommand{\P}{\text{Pr}}

\begin{document}

\title{Дорешивание}
\author{Антон Афанасьев}
\maketitle

\begin{enumerate}
\item[5.6] Распишем производящие функции.
$$e^z = \sum_{n=0}^\infty \frac{z^n}{n!}$$
$$\mathlarger{ \mathlarger{ e^{e^z} = \sum_{k=0}^\infty \frac{e^{kz}}{k!} = \sum_{k=0}^\infty \frac{\sum_{n=0}^\infty \frac{k^n z^n}{n!}}{k!}}}$$

Будем смотреть на это, как на экспоненциальную производящую функцию. Коэффициент при $\frac{z^n}{n!}$ равен $\sum_{k=0}^\infty \frac{k^n}{k!}$. 

Так как производящая функция для чисел Белла $B(z) = e^{e^z -1}$, то $eB(z) = e^{e^z}$, следовательно 
$$e \cdot B_n = \sum_{k=0}^\infty \frac{k^n}{k!}$$
$$B_n = \frac{1}{e}  \sum_{k=0}^\infty \frac{k^n}{k!}$$

\item[7.6] Будем строить помеченный покрашенный двудольный граф следующим образом. Разобьем вершины на два блока, в каждом из которых построим какой-то двудольный граф. В первом блоке в каждой компоненте связности вершину с минимальным номером покрасим в первый цвет (остальные вершины в компоненте красятся однозначно), а во втором блоке в каждой компоненте связности вершину с минимальным номером покрасим во второй цвет. Понятно, что таким действиям соответствует производящая функция $\widetilde{H}^2(z)$, так как мы по сути только разбили вершины на два блока и построили в них двудольные графы, дальше действия были однозначны. 

Покажем, что любой покрашенный двудольный граф (которые описываются функций $H(z)$) можно привести к такому же виду. Рассмотрим компоненты связности покрашенного двудольного графа. Если минимальная вершина в компоненте покрашена в первый цвет, то относим все вершины компоненты к первому блоку, если во второй, то относим все вершины компоненты ко второму блоку. Таким образом, мы построили биекцию между графами задающимися функциями $\widetilde{H}^2(z)$ и $H(z)$, следовательно $\widetilde{H}^2(z) = H(z)$.
\end{enumerate}
\end{document}
