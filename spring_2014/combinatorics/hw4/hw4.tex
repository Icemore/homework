\documentclass[10pt]{article}
\usepackage[russian]{babel}
\usepackage[utf8]{inputenc}
\usepackage{amssymb}
\usepackage{amsmath}
\usepackage{latexsym}
\usepackage{enumitem}
\usepackage[margin=2cm]{geometry}
\usepackage{relsize}

\newcommand{\rchoose}[2]{\left(\mkern-6mu \left({#1 \atop #2}\right) \mkern-6mu \right)}
\newcommand{\dsum}{\sum_{\substack{ k\le n \\ k - \text{нечетное}}}}
\renewcommand{\P}{\text{Pr}}

\begin{document}

\title{Домашняя работа 4}
\author{Антон Афанасьев}
\maketitle

\begin{enumerate}
	\item[7.2.] Всего орграфов на $n$ вершинах $g_n = 4 ^{\binom{n}{2}}$ (между каждой парой либо нет ребра, либо ребра в обе стороны, либо ребро в какую-то одну сторону). Пусть $G$ --- производящая функция для всех орграфов, $G^{(c)}$ --- производящая функция для слабо связных орграфов. Все орграфы могут быть получены следующим образом --- разбить множество вершин на какое-то количество блоков (компонент связности), в каждом блоке сделать слабо связный орграф. Поэтому производящие функции связаны отношением $G(z) = e^{G^{(c)}(z)}$.
	Пользуясь формулой (24) из конспекта получим:
	$$g_{n+1}^{(c)} = 4^{\binom{n+1}{2}} - \sum_{i=1}^n \binom{n}{i} 4^{\binom{i}{2}} \cdot g^{(c)}_{n+1-i}$$
	Первые несколько значений:
	
$g_{0}^{(c)} = 1$ \\
$g_{1}^{(c)} = 1$ \\
$g_{2}^{(c)} = 3$ \\
$g_{3}^{(c)} = 54$ \\
$g_{4}^{(c)} = 3834$ \\
$g_{5}^{(c)} = 1027080$ \\

	\item[7.4.] Заметим, что если мы проведем из первой вершины ребро в вершину $i+2$ пропустив $i$ вершин, то мы должны будем провести из каждой вершины ребро вперед пропуская $i$ вершин. Очевидно, что такие ``циклы'' с разным расстоянием $i$ либо не пересекаются (по ребрам), либо совпадают (если есть хоть одно ребро с расстоянием до следующей вершины $i$, то все остальные ребра тоже должны быть с таким же расстоянием). Поймем когда ``циклы'' совпадают. Ребро $(1, t)$ есть при расстоянии равном $t-2$. Но оно еще есть при расстоянии $n-t$ (теперь оно проводится ``вперед'' из $t$). Таким образом, $\lfloor n/2 \rfloor$ расстояний дают уникальный ``цикл''. Комбинируя их всеми возможными способами получим все звездные многоугольники, следовательно их число равно $2^{\lfloor n/2 \rfloor}$.
	
	\item[7.5.] Будем строить все турниры следующим образом --- разобьем множество вершин на какое-то количество блоков, в каждом блоке построим сильно связный турнир, а на блоках построим DAG, причем такой, что между каждой парой вершин есть ребро. Между вершинами в разных блоках проведем ребра в том направлении в каком соединяются блоки. Несложно видеть, что таким образом мы действительно получим все турниры, так как каждый из них разбивается на компоненты сильной связности, а на них задан ациклический граф. 
	
	Так как граф блоков не содержит циклов, то его вершины можно упорядочить в порядке топологической сортировки. Так как между каждой парой вершин есть ребро, то ребра в топологически упорядоченных вершинах проводятся единственным образом --- их каждой вершины во все вершины с большим номером. Таким образом, каждый такой DAG однозначно задается топологической сортировкой. Всего возможных топологических порядков $n$ вершин --- $n!$, а значит граф на блоках можно построить $n!$ способами. Производящая функция для этого действия $D(z) = \frac{1}{1 - z}$.
	
	Итак, \\
	$T(z) = D(T^c(z)) = \frac{1}{1 - T^c(z)}$ \\
	$T^c(z) = 1 - \frac{1}{T(z)}$
	
	\item[8.2.] Обозначим искомую производящую функцию как $T(z)$. Подвесим дерево за первую вершину. Заметим, что все вершины в дереве кроме корня одинаково устроены --- у нах либо нет детей (тогда это вершина степени 1, соединенная с родителем), либо у нее два ребенка (тогда это вершина степени 3). Обозначим производящую функцию перечисляющую поддеревья (начинающиеся не в корне) как $G(z)$. Тогда производящая функция для дерева на $n+1$ вершине $T^{r}(z) = G(z) + (G(z))^3$ (корень степени 1 или корень степени 3).
	
	$G(z) = z + z \cdot (G(z))^2$ \\
	$z = \frac{G(z)}{1 + (G(z))^2}$ \\
	$F(z) = \frac{z}{1+z^2}$\\
	$F(G(z)) = z$
	
	$g_n = \frac{1}{n} [ z^{-1} ] \frac{(1+z^2)^n}{z^n} = \frac{1}{n} [z^{n-1}] (1+z^2)^n$
	
	Так как $(1+z^2)^n = \sum_{i=0}^n \binom{n}{i} z^{2i}$, то
	
	$g_n = \frac{n!}{n} \binom{n}{(n-1)/2}$ при нечетном $n$ и 0 иначе.
	
	$$g_{2k+1} = \frac{(2k+1)!}{2k+1} \binom{2k+1}{k} = \frac{(2k+1)!}{2k+1} \cdot \frac{(2k+1)!}{k! \cdot (k+1)!} = \frac{2^k \cdot (2k+1)!! \cdot 2^k \cdot (2k+1)!!}{(2k+1) \cdot (k+1)} = \frac{4^k \cdot ((2k+1)!!)^2}{(2k+1) \cdot (k+1)}$$
	
\end{enumerate}
\end{document}
