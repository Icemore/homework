\documentclass[10pt]{article}
\usepackage[russian]{babel}
\usepackage[utf8]{inputenc}
\usepackage{amssymb}
\usepackage{amsmath}
\usepackage{latexsym}
\usepackage{enumitem}
\usepackage[margin=2cm]{geometry}
\usepackage{pseudocode}

\begin{document}

\title{Домашняя работа 9}
\author{Антон Афанасьев}
\maketitle

\begin{enumerate}

\item Подсчитаем при помощи алгоритма Дейкстры расстояния от вершины $s$ до всех остальных в графе и в дереве. Если для всех вершин они совпадают, то дерево является деревом кратчайших путей. Так как множество ребер в дереве является подмножеством множества ребер графа, то можно пройти от вершины $s$ до любой другой по этим ребрам и путь будет кратчайшим.

\item Мы разбирали эту задачу на практике. В поиске в ширину подсчитываем количество кратчайших путей для каждой вершины как сумму количества путей в уже посещенные соседние вершины, находящиеся на предыдущем уровне поиска в ширину.

\item Для каждой пары вершин $u, v$ кратчайший путь между ними, проходящий через $s$, состоит из двух частей, и выглядит как $u \leadsto s \leadsto v$. Каждая из частей должна быть кратчайшим путем (иначе весь путь не будет кратчайшим). Ответ на задачу --- сумма этих двух путей. Найдем алгоритмом Дейкстры кратчайшие пути из $s$ во все вершины, это будет вторая часть ответа. Найдем первую часть (пути из всех вершин в $s$), запустив алгоритм Дейкстры из вершины $s$ на транспонированном графу (у всех ребер меняется направление на противоположное).

Для разреженных графов ($E=O(V)$) будем использовать алгоритм Дейкстры с кучей, трудоемкость $O((|V| + |E|) \log |V|)$. Для плотных графов будем использовать алгоритм с массивом, трудоемкость $O(|V|^2)$.

\item Будем находить пути алгоритмом Дейкстры, только при обновлении текущих расстояний до вершин будем использовать нужные метрики. Так как добавление ребра к пути не может уменьшить вес пути по таким метрикам, алгоритм Дейкстры будет корректным.

\end{enumerate}
\end{document}
