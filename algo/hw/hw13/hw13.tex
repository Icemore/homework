\documentclass[10pt]{article}
\usepackage[russian]{babel}
\usepackage[utf8]{inputenc}
\usepackage{amssymb}
\usepackage{amsmath}
\usepackage{latexsym}
\usepackage{enumitem}
\usepackage[margin=2cm]{geometry}
\usepackage{pseudocode}

\begin{document}

\title{Домашняя работа 13}
\author{Антон Афанасьев}
\maketitle

\begin{enumerate}

\item[1] Заметим, что если у палиндрома убрать первый и последний символы (они очевидно одинаковые), то оставшаяся срока останется палиндромом. Используем это для сведения задачи к задаче меньшей размерности.

Динамика $d[k][l] = 1$, если в позиции $k$ заканчивается палиндром длины $l$, $0$, если нет. Начальные состояния: $d[k][1] = 1$, $d[k][0]=1$. Переход для $l > 1$:
$$d[k][l] = \begin{cases}
d[k-1][l-2],&\text{если }s[k]=s[k-l+1]\\
0,&\text{иначе}
\end{cases}
$$

Ответ: сумма $d[k][l]$ для $l \in \{1, \ldots n\}$, $k \in \{1, \ldots n\}$. 

\item[2] Пусть для строки $s$ у нас есть подсчитанная динамика из предыдущей задачи. Обозначим ее за $d_1$. Пусть новая динамика $d_2[k]$ --- минимальное количество палиндромов, на которое можно разбить префикс длины $k$. Начальное состояние: $d_2[0]=0$, $d_2[t]=\infty$ для $t \not = 0$. 

Для перехода переберем длину последнего палиндрома. Для длины $l$, если $d_1[k][l] = 1$ (действительно есть палиндром на этом месте), то релаксируем ответ: $d_2[k] = \min(d_2[k], d_2[k-l]+1)$. 

Для восстановления разбиения будем еще хранить для каждого $d_2[k]$ значение $l$ по которому мы получили оптимальный ответ. Тогда, откатываясь назад из последнего состояния по этим ссылкам мы восстановим разбиение.

Трудоемкость: $O(n^2)$ на подсчет первой динамике. $n$ состояний во второй и переход за $n$. Итого $O(n^2)$.

\item[4] Пусть вершины $n$-угольника занумерованы числами от 1 до $n$ в порядке обхода в какую-нибудь сторону. Рассмотрим грань, соединяющую вершины $n$ и $1$. Она должна принадлежать какому-то треугольнику триангуляции. Переберем какому. Этот треугольник разобьет наш $n$-угольник на два (или один) многоугольника, в которых нужно сделать то же самое.

Динамика $d[l][r]$ --- сумма весов диагоналей в оптимальной триангуляции многоугольника на вершинах $l, l+1, \ldots, r$. Начальное состояние: $d[i][i+1] = d[i][i+2] = 0$.\\
Переход:
$$d[l][r] = \min_{i \in \{l+1, \ldots, r-1\}} (w_{li} + w_{ri} + d[l][i] + d[i][r])$$
Предполагаем, что $w_{l, l+1} = 0$.

Ответ находится в $d[1][n]$. Для восстановления триангуляции будем хранить для каждого состояния $i$ из которого был получен оптимальный ответ. Проводя диагонали $(l,\ i)$, $(r,\ i)$ (если они диагонали) и запускаясь рекурсивно от половин проведем все диагонали, дающие оптимальную триангуляцию.

Состояний $O(n^2)$, переход за $O(n)$. Итого, трудоемкость $O(n^3)$.
\end{enumerate}
\end{document}
