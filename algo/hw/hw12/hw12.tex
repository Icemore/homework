\documentclass[10pt]{article}
\usepackage[russian]{babel}
\usepackage[utf8]{inputenc}
\usepackage{amssymb}
\usepackage{amsmath}
\usepackage{latexsym}
\usepackage{enumitem}
\usepackage[margin=2cm]{geometry}
\usepackage{pseudocode}

\begin{document}

\title{Домашняя работа 12}
\author{Антон Афанасьев}
\maketitle

\begin{enumerate}

\item Представим $A$ в системе счисления с основанием $c$: $A = a_0 + a_1 c + a_2 c^2 \ldots$. Такое разложение единственно. Так как нам разрешали использовать только монеты достоинства до $c^k$, наберем все коэффициенты при старших степенях при помощи $a_k$. Решение, которое берет $a_i$ монет достоинства $c^i$ --- оптимальное, так как при попытке набрать коэффициента при старших степенях при помощи младших увеличит ответ. Соответственно алгоритм: делим $A$ на $c^k$ пока можем, потом делим то что останется на $c^{k-1}$ и т.д.

\item Пусть у нас есть частоты символов $f_1 \le f_2 \le \ldots \le f_n$. Если $i<j$, то $ \frac{1}{2} < \frac{f_i}{f_j} \le 1$ и $ 1 \le \frac{f_j}{f_i} < 2$. 

$\frac{f_i}{f_n} + \frac{f_{i+1}}{f_n} > 1$ следовательно $f_i + f_{i+1} > f_n$. А значит, все вершины будут соединены в пары по порядку ($i$ с $i+1$). Получится $n/2 = 2^7$ узлов, с весами $f_i + f_{i+1}$. Покажем, что для них тоже выполняется условие, что частота любых двух символов различается меньше чем в два раза. Пусть $i<j$
$$\frac{f_i + f_{i+1}}{f_j + f_{j+1}} = \frac{1 + \frac{f_{i+1}}{f_i}}{\frac{f_j}{f_i} + \frac{f_{j+1}}{f_i}}$$
Т.к. $\frac{f_{i+1}}{f_i} \ge 1$, то $1 + \frac{f_{i+1}}{f_i} \ge 2$. Т.к. $\frac{f_j}{f_i} < 2$ и $\frac{f_{j+1}}{f_i} < 2$, то $\frac{f_j}{f_i} + \frac{f_{j+1}}{f_i} < 4$. Следовательно, вся дробь больше $\frac{1}{2}$, а значит вес узлов отличается не более чем в два раза.

На каждом уровне узлы будут объединяться парами, уменьшая количество элементов в два раза. Следовательно, в результате получится дерево высоты 8, с путем такой длины до каждого узла.

\item Отсортируем солдат противника во убыванию силы (если нам нельзя их двигать отсортируем индексы). Пусть массив солдат противника --- $a$, наших солдат --- $b$. Рассмотрим самого сильного солдата противника. Попробуем противопоставить ему нашего самого сильного солдата $b[1]$. Если он победит, то так его и оставим, если он проиграет, то поставим вместо него самого слабого нашего солдата $b[n]$. Такой выбор оптимальный, так как если врага не смог победить даже наш самый сильный солдат, то никакой наш солдат его не победит. Поэтому нам выгоднее отдать самого слабого, так как он стоит меньше всего очков: замена $b[n]$ на $b[j]$, проигрывающего $a[i]$,  ничего не изменит, так как $b[n] < b[j] < a[i]$ ($b[n]$ будет проигрывать), а на выигрывающего $b[j]$ менять не выгодно, так как даже если $b[n]$ победит он принесет меньше очков. Если же сильный побеждает, то менять его нет смысла, так как в любом другом месте он победит, и, если какой-то наш солдат $b[j]$ будет проигрывать $a[i]$, то его замена на $b[0]$ ничего не изменит, так как $b[j]$ теперь будет проигрывать $a[0]$.

Будем продолжать так до конца массива $a$, поддерживая указатели слева и справа на текущих самого сильного и слабого бойцов.\\
Трудоемкость: $O(n \log n)$ на сортировку.

\end{enumerate}
\end{document}
