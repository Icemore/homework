\documentclass[10pt]{article}
\usepackage[russian]{babel}
\usepackage[utf8]{inputenc}
\usepackage{amssymb}
\usepackage{amsmath}
\usepackage{latexsym}
\usepackage{enumerate}
\usepackage[margin=2cm]{geometry}

\begin{document}

\title{Домашняя работа 3}
\author{Антон Афанасьев}
\maketitle

\begin{enumerate}[1.]
\item Асимптотика\\
$T(n) = aT(\frac{n}{b}) + n^d$

\begin{enumerate}[(a)]
\item 
	$T(n) = T(\lceil \frac{n}{2} \rceil) +1$\\
	$\log_b a = \log_2 1 = 0 = d$\\
	$\Rightarrow T(n)=O(\log n)$

\item
	$T(n) = (T \cdot \alpha) + T((1 - \alpha) \cdot n) + n$\\
	Представим рекурсию в виде дерева. На каждом уровне выполняется $n$ операций, уровней $-\log_\alpha n$.\\
	$T(n) = O(n \log n)$
	
\item
	$T(n) = 4 \cdot T( \lfloor \frac{n}{2} \rfloor) + n^k$\\
	$\log_b a = \log_2 4 = 2$\\
	$k=1 : O(n^2)$\\
	$k=2 : O(n^2 \cdot \log n)$\\
	$k=3 : O(n^3)$
	
\item
\item 
	$T(b) = 2 \cdot T(\lfloor \frac{n}{2} \rfloor + 17) + n$\\
	Представим рекурсию в виде дерева. Высота дерева --- $\log n$, на $k$-ом уровне $2^k$ вершин. На уровне $k$ в каждой вершине выполняется операций:
 	$$\frac{n + 17 \cdot \sum_{i=1}^k 2^i}{2^k} = \frac{n +17(2^{k+1} -2)}{2^k}$$
 	А значит всего на уровне $k$ выполняется $n+17(2^{k+1}-2)$ операций.
 	$$T(n) = \sum_{k=0}^{\lfloor \log n \rfloor} n+17(2^{k+1}-2) \le n\log n + 17 \sum_{k=0}^{\lfloor \log n \rfloor} 2^{k+1} - 34\log n \le n\log n + 34 \cdot 2^{\log n} - 34 \log n = n\log n + 34 n - 34 \log n$$
 	$T(n) = O(n \log n)$
 	
\item 
	$T(n) = 2 \cdot T( \lfloor \log n \rfloor ) + 2^{\log^* n}$
	Представим рекурсию в виде дерева. Высота дерева --- $\log^* n$. На уровне $k$: $2^k$ вершин, в каждой вершине $2^{\log^*(n) - k}$ операций. Всего на уровне $k$ операций -- $2^{\log^* n}$.\\
	$T(n) = O(\log^*( n) \cdot 2^{\log^* n})$
\end{enumerate}
\item
\item Сдал на паре
\item Даны отсортированные массивы $a_0, a_1, \ldots, a_{n-1}$ и $b_0, b_2, \ldots, b_{m-1}$. Найти элемент, у которого будет индекс $k$ в их отсортированном объединении.

	Пусть $a_i$ --- ответ. Тогда $b_{k-i} \ge a_i$ и $b_{k-i-1} \le a_i$, причем это выполняется только для ответа (перед $a_i$ в объединенном массиве будут стоять только элементы $a_j, j<i$ и $b_t, t<k-i$. $a_i$ будет иметь индекс $k$). Здесь и далее будем считать, что есть элементы $b_{-1} = -\infty$ и $b_m=\infty$.
	
	Будем искать $a_i$ бинарным поиском. Начальные границы $l=0$, $r=min(k, n)$. На каком-то шаге $i=(l+r)/2$.\\
	Если условия выполняются --- мы нашли ответ.\\
	Если $b_{k-i} < a_i$, то $r=i-1$.\\
	Если $b_{k-i-1} > a_i$, то $l=i+1$.
	
	Если мы не найдем ответ в первом массиве произведем такой же поиск во втором.
	
	Таким образом, мы найдем ответ не более чем за два бинпоиска и сложность алгоритма $O(\log n + \log m)$.
\end{enumerate}
\end{document}
