\documentclass[10pt]{article}
\usepackage[russian]{babel}
\usepackage[utf8]{inputenc}
\usepackage{amssymb}
\usepackage{amsmath}
\usepackage{latexsym}
\usepackage{enumerate}
\usepackage[margin=2cm]{geometry}

\begin{document}

\title{Домашняя работа 2}
\author{Антон Афанасьев}
\maketitle

\begin{itemize}

\item Скобочная последовательность 

В корректной скобочной последовательности всегда существует пара скобок которые стоят непосредственно друг за другом (самые ``внутренние'' скобки), причем если их удалить скобочная последовательность останется корректной. Будем по очереди удалять такие скобки, пока можем. Если в итоге ничего не останется, то скобочная последовательность была корректной.\\
Для этого используем стек. Идем по входной строке слева направо. Для каждого символа:
\begin{itemize}
\item Если это открывающая скобка просто кладем ее в стек.
\item Если это закрывающая скобка, то достаем из стека ей парную. Если стек пуст, или на его вершине скобка другого типа, скобочная последовательность нарушена.
\end{itemize}
Если после такого прохода стек пуст, то скобочная последовательность корректна.

\item Сумма на отрезке

Предподсчитаем частичные суммы на префиксах $C_i = \sum_{j=0}^i a_j$. Это можно сделать за линейное время, так как \\
$C_0=a_0$\\
$C_i = C_{i-1} + a_i$

Теперь можно отвечать на запрос за константу, т. к.
$$\sum_{i=l}^r a_i = C_r - C_{l-1}$$

\item Вычеркивание цифр\\
Дано число c цифрами $a_1, a_2, \ldots, a_n$.

Максимизируем первую цифру искомого числа, т.к. если она не будет наибольшей возможной, то все число не будет максимальной. На первое место можно поставить любую цифру из $a_1, \ldots, a_{k+1}$ (максимум можно вычекнуть первые $k$ цифр). Возьмем из них максимальную, а из них самую левую, пусть это будет $a_i$. Останется такая же задача меньшей размерности - найти максимальное число, которое получится из $a_{i+1}, \ldots, a_n$ вычеркиванием $k-i$ цифр. Лидирующие нули нас здесь не волнуют, так как в качестве первой цифры мы его точно не выберем (выбираем максимальную цифру, а нам гарантировали, что $a_1$ не 0), а остальные цифры ничего не испортят.


\end{itemize}

\end{document}
