\documentclass[10pt]{article}
\usepackage[russian]{babel}
\usepackage[utf8]{inputenc}
\usepackage{amssymb}
\usepackage{amsmath}
\usepackage{latexsym}
\usepackage{enumitem}
\usepackage[margin=2cm]{geometry}
\usepackage{pseudocode}

\begin{document}

\title{Домашняя работа 7}
\author{Антон Афанасьев}
\maketitle

\begin{enumerate}

\item Запустим поиск в глубину, будем красить вершины в два цвета --- каждую красим в цвет, отличный от цвета вершины из которой мы пришли. Теперь в дереве поиска в глубину каждое ребро соединяет две вершины разного цвета. Т.к. дерево покрывает все вершины, то у каждой вершины есть такое ребро, а значит у каждой вершины есть сосед в другой доле.

\item Пусть вершина --- задача, ребро --- отношение ``зависит''. Определим для каждой вершины минимальное время, через которое ее можно завершить. Будем перебирать вершины в порядке, обратном топологической сортировке. Ответ $T_v$ для вершины $v$:
$$T_v = \max_{(v, u) \in E}(T_u) + t_v$$
Т.к. мы рассматривали вершины в порядке, обратном топологической сортировке, $T_u$ уже известны. Ответ для всего проекта --- максимальное $T_v$ по всем вершинам.

\item Подвесим дерево за вершину, в которой изначально стоит фишка. После хода игра будет продолжаться только в том поддереве, в которое мы пошли. Получаем такую же задачу меньшего размера, только теперь второй игрок ходит первым. Назовем вершину выигрышной, если игрок который ходит первым из этой вершины выигрывает при оптимальной игре, и проигрышной, если он проигрывает. Листья --- проигрышные вершины. Для остальных вершин:
	\begin{itemize}
	\item вершина выигрышная, если из нее есть ход в проигрышную
	\item вершина проигрышная, если из нее все ходы в выигрышные вершины
	\end{itemize}
Разметим вершины таким образом поиском в глубину, запустившись от корня. Если корень --- выигрышная вершина, то выигрывает тот, кто ходит первым, иначе другой.

\item[5.] Попробуем покрасить граф в два цвета. Если в нем есть цикл нечетной длины нам это не удастся. Будем поиском в глубину красить вершины в два цвета и запоминать для каждой вершины откуда мы пришли. Когда поиск в глубину найдет нарушение раскраски (ребро которое ведет в вершину такого же цвета), восстановим цикл по сохраненным ссылкам.
\end{enumerate}
\end{document}
